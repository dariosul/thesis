\documentclass [t] {beamer} % KOMA script (article)

\mode<presentation> {
%\usetheme{boadilla}
%\usetheme{default}
%\usetheme{AnnArbor}
%\usetheme{Antibes}
%\usetheme{Bergen}
%\usetheme{Berkeley}
%\usetheme{Berlin}
%\usetheme{Boadilla}
%\usetheme{CambridgeUS}
%\usetheme{Copenhagen}
%\usetheme{Darmstadt}
%\usetheme{Dresden}
%\usetheme{Frankfurt}
%\usetheme{Goettingen}
%\usetheme{Hannover}
%\usetheme{Ilmenau}
%\usetheme{JuanLesPins}
%\usetheme{Luebeck}
\usetheme{Madrid}
%\usetheme{Malmoe}
%\usetheme{Marburg}
%\usetheme{Montpellier}
%\usetheme{PaloAlto}
%\usetheme{Pittsburgh}
%\usetheme{Rochester}
%\usetheme{Singapore}
%\usetheme{Szeged}
%\usetheme{Warsaw}

%\usecolortheme{albatross}
%\usecolortheme{beaver}
%\usecolortheme{beetle}
%\usecolortheme{crane}
%\usecolortheme{dolphin}
%\usecolortheme{dove}
%\usecolortheme{fly}
%\usecolortheme{lily}
%\usecolortheme{orchid}
\usecolortheme{rose}
%\usecolortheme{seagull}
%\usecolortheme{seahorse}
%\usecolortheme{whale}
%\usecolortheme{wolverine}

%\setbeamertemplate{footline} % To remove the footer line in all slides uncomment this line
%\setbeamertemplate{footline}[page number] % To replace the footer line in all slides with a simple slide count uncomment this line

%\setbeamertemplate{navigation symbols}{} % To remove the navigation symbols from the bottom of all slides uncomment this line
}

%\usepackage{beamerthemesplit}
\usepackage{appendixnumberbeamer}
%\usepackage{biblatex}
%\addbibresource{xampl.bib}

\usepackage{graphicx} % Allows including images
\usepackage{booktabs} % Allows the use of \toprule, \midrule and \bottomrule in tables
\usepackage{lipsum}
%\newcommand\fontup{\fontsize{20}{7.2}\selectfont}

\usepackage{amsmath}
%\interdisplaylinepenalty=2500
\usepackage{array}
%\hyphenation{op-tical net-works semi-conduc-tor}


\newcommand{\symb}[1]{\mbox{\boldmath${#1}$}}
\newcommand{\Ba}{\mbox{\boldmath$a$}}
\newcommand{\Bb}{\mbox{\boldmath$b$}}
\newcommand{\Bh}{\mbox{\boldmath$h$}}
\newcommand{\Bd}{\mbox{\boldmath$d$}}
\newcommand{\Be}{\mbox{\boldmath$e$}}
\newcommand{\Bg}{\mbox{\boldmath$g$}}
\newcommand{\Bf}{\mbox{\boldmath$f$}}
\newcommand{\Bi}{\mbox{\boldmath$i$}}
\newcommand{\Bj}{\mbox{\boldmath$j$}}
\newcommand{\Bp}{\mbox{\boldmath$p$}}
\newcommand{\Bq}{\mbox{\boldmath$q$}}
\newcommand{\Br}{\mbox{\boldmath$r$}}
\newcommand{\Bs}{\mbox{\boldmath$s$}}
\newcommand{\Bu}{\mbox{\boldmath$u$}}
\newcommand{\Bv}{\mbox{\boldmath$v$}}
\newcommand{\Bz}{\mbox{\boldmath$z$}}
\newcommand{\Bx}{\mbox{\boldmath$x$}}
\newcommand{\By}{\mbox{\boldmath$y$}}
\newcommand{\Beps}{\symb{\varepsilon}}
\newcommand{\BA}{\mbox{\boldmath$A$}}
\newcommand{\BC}{\mbox{\boldmath$C$}}
\newcommand{\BB}{\mbox{\boldmath$B$}}
\newcommand{\BD}{\mbox{\boldmath$D$}}
\newcommand{\BF}{\mbox{\boldmath$F$}}
\newcommand{\BG}{\mbox{\boldmath{$G$}}}
\newcommand{\BGA}{\mbox{\boldmath{$\Gamma$}}}
\newcommand{\BLA}{\mbox{\boldmath{$\Lambda$}}}
\newcommand{\BH}{\mbox{\boldmath$H$}}
\newcommand{\BU}{\mbox{\boldmath$U$}}
\newcommand{\BV}{\mbox{\boldmath$V$}}
\newcommand{\BO}{\mbox{\boldmath{$0$}}}
\newcommand{\BW}{\mbox{\boldmath$W$}}
\newcommand{\BI}{\mbox{\boldmath$I$}}
\newcommand{\BX}{\mbox{\boldmath$X$}}
\newcommand{\BP}{\mbox{\boldmath$P$}}
\newcommand{\BQ}{\mbox{\boldmath$Q$}}
\newcommand {\Min} {\mathop{\mbox{minimize}}}
\newcommand {\Max} {\mathop{\mbox{maximize }}}
\newcounter{abc}
\renewcommand{\theequation}{\arabic{equation}\alph{abc}}
\newcommand{\ome}{(\omega_1, \omega_2)}

\setbeamercolor{bibliography entry author}{fg=black}
 
%----------------------------------------------------------------------------------------
%	TITLE PAGE
%----------------------------------------------------------------------------------------

\title[Localization in PSN]{Localization Algorithms in Passive Sensor Networks} 

\author{Darya Ismailova} 
\institute[UVic] 
{
Department of Electrical and Computer Engineering \\
University of Victoria, Victoria, BC, Canada \\
\medskip
%\textit{ismailds@uvic.ca, wslu@uvic.ca} % Your email address
}
\date{January 19, 2017}

\begin{document} 
%\fontsize{14}{14}\selectfont

\begin{frame} % Slide 1
\titlepage 
\end{frame}


%----------------------------------------------------------------------------------------
%	PRESENTATION SLIDES
%----------------------------------------------------------------------------------------

%------------------------------------------------

\section{Introduction} 

\begin{frame} % Slide 2
\frametitle{Outline} 
\begin{enumerate}
\item
Motivation
\\~\\
\item
Basic Localization Systems and Methods
\\~\\
\item
Iterative Re-Weighting Least-Squares Methods for Source Localization
\\~\\
\item
Penalty Convex-Concave Procedure for Range-based Localization
\\~\\
\item
Conclusions and Future Work
\end{enumerate}
\end{frame}


\begin{frame} % Slide 3
\frametitle{Introduction}
\phantom{m}
\begin{itemize}
\item
Navigation: outdoor; indoor
\\~\\
\item
Surveillance
\\~\\
\item
Localization of emergency callers
\\~\\
\item
Emergency and rescue operations / first responders
\\~\\
\item
Self-organizing networks
\\~\\
\item
Asset monitoring and tracking
\\~\\
\item
Other commercial location-based servises
\\~\\
\item
\ldots
\end{itemize}
\end{frame}

\begin{frame} % Slide 4
\frametitle{Introduction}
\phantom{m}
\begin{itemize}
\item
Ranging methods
\begin{itemize}
\item
range measurements (Time Of Arrival)
\item
range-difference measurements (Time-Difference of Arrival)
\item
received signal strength
\\~\\
\end{itemize}

\item
Angle Of Arrival Techniques
\\~\\
\item
Survey-Based Systems (fingerprinting)
\begin{itemize}
\item
memoryles systems (SVM, NN)
\item
memory systems (Bayesian inference, grid-based Markov)
\item
channel impulse response fingerprinting
non-RF features
\end{itemize}
\end{itemize}
\end{frame}

\begin{frame} % Slide 5
\frametitle{Basic Localization Systems and Methods}
\begin{figure}[h]
%\centering
\includegraphics[width=1.0\textwidth]{../figures/localization_example.png}
\caption{Classical geolocation system. Range or angle information is extracted from received RF signals. Location is then estimated by lateration/angulation techniques [GeoLoc].}
\label{fig:2step}
\end{figure}
\end{frame}

\begin{frame} % Slide 6
\frametitle{Time Of Arrival Localization (TOA)}
\begin{figure}[h]
\includegraphics[height=0.6\textheight]{../figures/toa_example.png}
\caption{TOA-based trilateration. Range measurements to at least three BS make
up a set of nonlinear equations that can be solved to estimate the position of a
signal source [GeoLoc].}
\label{fig:2step}
\end{figure}
\end{frame}

\begin{frame} % Slide 7
\frametitle{Time Of Arrival Localization (TOA)}
\phantom{m}
The nonlinear least squares (NLLS) source location extimate $\hat{\Bx}$ is found by
\begin{equation}
\nonumber
\hat{\Bx} = \mbox{argmin}_{\Bx}\left\lbrace \sum_{i=1}^m \beta_i\left(d_n^i - \|\Bx -  \Ba_i \| \right)^2 \right\rbrace
\end{equation}
where 
\\
$\Ba_i$- a vector of known coordinates of reference points (sensors)
\\~\\
$d_n^i $ - a noisy range measurement associated with it
\\~\\
$\beta_i$ - a weight used to emphasize the degree of confidence in the
measurement
\\~\\
$m$ - the number of sensors.
\end{frame}

\begin{frame} % Slide 8
\frametitle{Time-Difference Of Arrival Localization (TDOA)}
\begin{figure}[h]
\includegraphics[height=0.7\textheight]{../figures/tdoa_example.png}
\caption{Example of observed time-difference of arrival (O-TDOA) method [GeoLoc].}
\label{fig:2step}
\end{figure}
\end{frame}

\begin{frame} % Slide 9
\frametitle{Time-Difference Of Arrival Localization (TDOA)}
\phantom{m}
Given the range-difference measurements
\begin{equation}
\nonumber
d_i = \|\Bx -  \Ba_i \| - \|\Bx -  \Ba_0 \| = \|\Bx -  \Ba_i \| - \|\Bx\|, \mbox{for } i = 1, 2, \ldots, m
\end{equation}
The standard NLLS location estimate $\hat{\Bx}$ is found by
\begin{equation}
\nonumber
\hat{\Bx} = \Min_{\Bx} \sum_{i=1}^m \left(\|\Bx -  \Ba_i \| - \|\Bx\| - d_n^i\right)^2
\end{equation}
with 
\\
$\Ba_i$- a vector of known coordinates of reference points (sensors)
\\
$d_n^i $ - a noisy range-difference measurement associated with it
\\
$m$ - the number of sensors.
\end{frame}



\begin{frame} % Slide 10
\frametitle{Methods Based on Received Signal Strength (RSS-based)}
\phantom{m}
The relationship between the RSS reading and the distance can be
approximated by
\begin{equation}
\nonumber
P_x(d) = P_0(d_0) - 10n_p\log_{10}\left( \frac{d_i}{d_0}\right) + X_{\sigma}
\end{equation}
where
\\
$P_0(d_0)$ - a reference power in dB milliwatts at a reference distance $d_0$ away from the transmitter
\\
$n_p$ - the pathloss exponent
\\
$X_{\sigma}$ - the log-normal shadow fading component with variance $\sigma^2$
\\
$d_i$ - the distance between the mobile devices and the ith base station
\\
$\sigma$ and $n_p$ are environment dependent
\end{frame}


\begin{frame} % Slide 11
\frametitle{Why Least Squares}
\phantom{m}
\begin{itemize}
\item
Least squares (LS) algorithms for range-based localization:\\
- geometrically meaningful\\
- provide low complexity solutions with competitive accuracy
\\~\\
\item
However:\\
- the error measure is non-convex\\
- excludes many local methods, that are iterative
\\~\\
\item
Solutions obtained using global localization techniques such as
semidefinite programming (SDP) are not optimal in LS sense.
\end{itemize}
\end{frame}


\begin{frame} % Slide 12
\frametitle{Iterative Re-Weighting Least-Squares Methods for Source
Localization}
\phantom{m}
\begin{itemize}
\item
Methods developed by A. Beck, P. Stoica, J. Li [BSL2008] for \textit{squared} range LS (SR-LS) and \textit{squared} range difference LS (SDR-LS) problems allow to obtain exact and \textit{global}  solutions.
\\~\\
\item
The results produced are merely approximations of the original LS
problems because SR-LS and SRD-LS are no longer an ML solutions.
\\~\\
\item
Proposed iterative procedure where the SR-LS (or SRD-LS) algorithm is applied to a \textit{weighted} sum of squared terms and special weights construction allow to obtain a solution which is conciderably closer to
the original range-based (or range-difference-based) LS solution.
\end{itemize}
\end{frame}




\section[Chapter 2]{Source Localization From Range Measurements}

\subsection{Measurement Model} %2.2

\begin{frame} % Slide 13
\frametitle{Source Localization From Range Measurements}
{\large \textit{Measurement Model}} \\~\\
\normalsize
\begin{itemize}
\item 
Throughout it is assumed that \textit{range measurements} obey the model
\begin{equation} 
\nonumber
%\setcounter{equation}{1}
r_i = \|{\Bx} - {\Ba}_i\| + \varepsilon_i, \quad i = 1, \ldots , m.
\end{equation}  
where $\{\Ba_1,\ldots, \Ba_m\}$ - given array of $m$ sensors;\\
$\Ba_i\in R^n$  contains $n$ coordinates of the $i$th sensor in space $R^n$; \\
$r_i$ - received noisy distance reading from the $i$th sensor; \\
$\varepsilon_i$ - unknown noise associated with measurement from the $i$th sensor. 
\\~\\
\item
The problem can be stated as to estimate the exact source location $\Bx \in R^n$ from noisy range measurements $\Br = [r_1 \ r_2 \ldots r_m]^T$.
\end{itemize}
\end{frame}



\begin{frame} % Slide 14
\frametitle{Source Localization From Range Measurements}
{\large \textit{LS Formulations}} \\~\\
\normalsize
\begin{itemize}
\item 
The range-based least squares (R-LS) estimate refers to the solution of the problem
\begin{equation} \label{R} 
\Min_{\Bx} f(\Bx)=\sum_{i=1}^{m} (r_i - \|{\Bx} - {\Ba}_i\|)^2	\tag{R}
\end{equation} \\
\item If $\Beps \sim N(0,\symb{\Sigma}) $ and $\symb{\Sigma} \propto \BI$, then the R-LS solution of problem \eqref{R} is identical to the ML location estimator. \\ ~\\
%This formulation is connected to the ML estimator that determines source location by examining the probabilistic model of  $\Beps$. \\~\\
 \item  Unfortunately, the objective in \eqref{R} is highly non-convex, posessing many local minimizers even for small-scale systems.
\end{itemize}
\end{frame}



\begin{frame}
\frametitle{Source Localization From Range Measurements}
{\large \textit{LS Formulations}} \\~\\
\normalsize
\begin{itemize}
\item 
Alternatively, location estimate can be obtained by solving the \textit{squared range based LS} (SR-LS) problem [BSL2008]
\begin{equation} \label{SR}
\Min_{\Bx} \sum_{i=1}^{m} (\|{\Bx} - {\Ba}_i\|^2 - r_i^2)^2 \tag{SR}
\end{equation}

\item  
The SR-LS estimate is no longer an ML solution, hence, only an approximation of the original R-LS problem.\\
\item 
To reduce the gap between the two solutions we propose a weighted SR-LS (WSR-LS) problem:
\begin{equation} \label{WSR}
\Min_{\Bx} \sum_{i=1}^{m} w_i\left(\|{\Bx} - {\Ba}_i\|^2 - r_i^2\right)^2 \tag{WSR}
\end{equation}
\end{itemize}
\end{frame}



\begin{frame} % Slide 16
\frametitle{Source Localization From Range Measurements}
{\large \textit{An Iterative Re-Weighting Strategy}} \\~\\
\normalsize
\begin{itemize}
\item 
WSR-LS with properly chosen weights facilitates an excellent approximation of the R-LS estime. \\~\\

\item 
The main idea is to use the weigths ${w_i, i = 1, \ldots, m}$ to tune the objective in \eqref{WSR} toward the objective in \eqref{R}. % \\~\\
%\item
%To substantiate the idea, 
%To see this we 
%We compare the $i$th term of the objective in \eqref{WSR} with its counterpart in \eqref{R} as:
\begin{equation} 
\nonumber
\underbrace{w_i\left(\|\Bx-\Ba_i\|^2-r_i^2\right)^2}_{\mbox{in (WSR)}}\leftrightarrow\underbrace{\left(\|\Bx-\Ba_i\|-r_i\right)^2}_{\mbox{in (R)}}
\end{equation}
\end{itemize}
\end{frame}



\begin{frame} % Slide 17
\frametitle{Source Localization From Range Measurements}
{\large \textit{An Iterative Re-Weighting Strategy}} \\~\\
\normalsize
\begin{itemize}
\item 
By writing the $i$th term in \eqref{WSR} as
\begin{equation}
\nonumber
\begin{array}{l}
w_i\left(\|\Bx-\Ba_i\|^2-r_i^2\right)^2 = w_i\left(\|\Bx-\Ba_i\|+r_i\right)^2 \underbrace{\left(\|\Bx-\Ba_i\|-r_i\right)^2}_{\mbox{same as in (R)}}
\end{array}
\end{equation}
we note that the objective in \eqref{WSR} would be the same as in \eqref{R} if the weight $w_i$ was assigned to $1/\left(\|\Bx-\Ba_i\|+r_i\right)^2$. \\~\\
\item 
Evidently, such weight assignments cannot be realized.
\end{itemize}
\end{frame}


\begin{frame}% Slide 17
\frametitle{Source Localization From Range Measurements}
{\large \textit{An Iterative Re-Weighting Strategy}} 
\\~\\
\normalsize

\begin{itemize}
\item 
 In the proposed iterative procedure we solve a weighted SR-LS sub-problem,where at each iteration the weights are fixed:
\begin{equation} \label{IRWSR}
\Min_{x} \sum_{i=1}^m w_i^{(k)}\left(\|\Bx-\Ba_i\|^2-r_i^2\right)^2 \tag{IRWSR}
\end{equation}
\item 
 for $k=1$ all weights $\{w_i^{(1)}, i=1,\ldots, m\}$ are set to unity; \\
 \item 
 for $k\geq2$ the weights $\{w_i^{(k)},i=1,\ldots,m\}$ are assigned using the previous iterate $\Bx_{k-1}$ as
\begin{equation} 
\nonumber
w_i^{(k)}=\frac{1}{\left(\|\Bx_{k-1}-\Ba_i\|+r_i\right)^2}.
\end{equation}

\end{itemize}
\end{frame}



\section[SRD-LS Methods]{Source Localization From Range-Difference Measurements}
\subsection{Problem Statement} %2.2

%------------------------------------------------

\begin{frame} % Slide 19
\frametitle{Source Localization From Range-Difference Measurements} %3.2
{\large \textit{Problem Statement}}
\\~\\
\normalsize
\begin{itemize}
\item 
It is assumed that the range-difference measurements obey the model:
 \begin{equation} 
 \nonumber
 d_i=\|\Bx-\Ba_i\|-\|\Bx-\Ba_0\|=\|\Bx-\Ba_i\|-\|\Bx\|, \quad i = 1,\ldots,m
 \end{equation}
where $\Ba_0$ - reference sensor placed at the origin.\\~\\
 \item 
 The standard range-difference LS (RD-LS) problem is formulated as
 \begin{equation} \label{RD}
\Min_{\Bx \in R^n} F(\Bx)=\sum_{i=1}^m \left(d_i+\|\Bx\|-\|\Bx-\Ba_i\|\right)^2 \tag{RD}
 \end{equation}
 \end{itemize}
\end{frame}



\begin{frame} % Slide 20
\frametitle{Source Localization From Range-Difference Measurements} %3.2
{\large \textit{SRD-LS and WSRD-LS formulations}}
\\~\\
\normalsize
\begin{itemize}
\item 
An approximation of the RD-LS solution can be obtained by solving the \textit{squared range difference based LS} (SRD-LS) problem. \\ % [BSL2008]. \\
\item 
By re-writing the measurements model as $d_i+\|\Bx\|=\|\Bx-\Ba_i\|$ and squaring both sides, we obtain
  \begin{equation} 
  \nonumber
-2d_i\|\Bx\|-2\Ba_i^T\Bx=g_i, \quad i=1,\ldots,m
 \end{equation}
 where $g_i=d_i^2-\|\Ba_i\|^2$.  
The SRD-LS solution can be obtained by minimizing following criterion:
 \begin{equation}
 \nonumber
\Min_{{\Bx} \in R^{n}} \sum_{i=1}^m \left(-2\Ba_i^T\Bx-2d_i\|\Bx\|-g_i\right)^2 
\end{equation}

 \end{itemize}
\end{frame}


\begin{frame} % Slide 21
\frametitle{Source Localization From Range-Difference Measurements} 
{\large \textit{Improved Solution Using Iterative Re-weighting}} \\~\\
\normalsize
\begin{itemize}
\item 
We now present a method for improved solutions over SRD-LS solutions. 
\\~\\
\item 
We consider the weighted SRD-LS problem
\begin{equation} \label{WSRD}
\Min_{{\Bx} \in R^n} \sum_{i=1}^m w_i\left(-2\Ba_i^T\Bx-2d_i\|\Bx\|-g_i\right)^2 \tag{WSRD}
\end{equation}
where weights $w_i$ for $i=1,\ldots,m$ are \textit{fixed} nonnegative constants. 
\end{itemize}
\end{frame}



\begin{frame} % Slide 22
\frametitle{Source Localization From Range-Difference Measurements} %3.2
{\large \textit{Improved Solution Using Iterative Re-weighting}} 
\\~\\
\normalsize
\begin{itemize}
\item <1->
The $i$th term of the objective function in \eqref{WSRD} can be written as:
\begin{equation}
\nonumber
\begin{aligned}
&w_i\left(-2d_i\|\Bx\|-2\Ba_i^T\Bx-g_i\right)^2 \\
%= &w_i\left((d_i+\|\Bx\|)^2-\|\Bx-\Ba_i\|^2\right)^2 \\
=&w_i\left(d_i+\|\Bx\|+\|\Bx-\Ba_i\|\right)\underbrace{\left(d_i+\|\Bx\|-\|\Bx-\Ba_i\|\right)}_{\mbox{same as in RD}}
\end{aligned}
\end{equation}
\\

\phantom{m} 
\item 
If weights $w_i$ were set to $1/\left(d_i+\|\Bx\|+\|\Bx-\Ba_i\|\right)^2$  the objective in \eqref{WSRD} would be the same as in \eqref{RD}. 
\end{itemize}
\end{frame}



\begin{frame} % Slide 23
\frametitle{Source Localization From Range-Difference Measurements} %3.2
{\large \textit{Improved Solution Using Iterative Re-weighting}} 
\\~\\
\normalsize
\begin{itemize}
\item 
We employ an iterative procedure  where the weights in the $k$th iteration are assigned to 
\begin{equation} 
\nonumber
w_i^{(k)}=\frac{1}{\left(d_i+\|\Bx_{k-1}\|+\|\Bx_{k-1}-\Ba_i\|\right)^2}, i=1,\ldots,m
\end{equation} \\~\\
with $\{w_i^{(1)} = 1, i=1,\ldots, m\}$.
 \\~\\
\item 
We will refer to the derived problem as the iterative re-weighted SRD-LS (WSRD-LS) problem and the solution obtained as IRWSRD-LS solution. 

\end{itemize}
\end{frame}



\begin{frame} % Slide 24

\frametitle{Performance Evaluation for SR-LS and IRWSR-LS} 
\normalsize
\begin{itemize}
\item 
We can see that IRWSR-LS solutions offer considerable improvement over SR-LS solutions.
\begin{table}
\caption[c]{Averaged MSE for SR-LS and IRWSR-LS methods by noise level}
\begin{tabular}{|c|c|c|c|}
\toprule
\textbf{$\sigma$ } & \textbf{SR - LS} & \textbf{IRWSR-LS } & \textbf{Improvement (\%)}\\
\midrule 
1e-03&	1.897294e-06&	1.123411e-06& 40.8	\\ &&&\\
1e-02&	1.779870e-04&	1.081470e-04& 39.2	\\ &&&\\
1e-01&	1.831870e-02&	1.128165e-02& 38.4	\\ &&&\\
1e+0&	2.415438e+00&	1.877930e+00& 22.3	\\ %&&&&\\
\bottomrule
\end{tabular}
\end{table}

\end{itemize}
\end{frame}


\begin{frame} % Slide 25
%\frametitle{An Iterative Re-Weighting Strategy}
\frametitle{Performance Evaluation for SRD-LS and IRWSRD-LS} 
\normalsize

\begin{table}
\caption{Averaged MSE for SRD-LS and IRWSRD-LS methods by noise level}
\begin{tabular}{|c|c|c|c|}
\toprule
\textbf{$\sigma$ } & \textbf{SRD - LS} & \textbf{IRWSRD-LS } & \textbf{Improvement (\%)}\\
\midrule
1e-04&	8.4918e-09&	4.1050e-09& 51.7\\ &&&\\
1e-03&	5.8553e-06&	3.5105e-06& 40.0\\ &&&\\
1e-02&	6.3508e-05&	5.0378e-05& 20.7\\ &&&\\
1e-01&	1.6057e-02&	1.0055e-02& 37.3\\ &&&\\
1e+0&	1.2773e+00&	6.2221e-01& 51.2\\ %&&&&\\
\bottomrule
\end{tabular}
\end{table}
\end{frame}

\begin{frame} % Slide
\frametitle{}
\phantom{m}
\begin{itemize}
\item

\end{itemize}
\end{frame}

%%%%%%%%%%%%%
%%%b IRW
%%%%%%%%%%%%%

% A.1
\begin{frame}
\frametitle{Nonconvexity of the R-LS objective}
\phantom{m}
\linespread{0.1} \selectfont
Given the objective
\begin{equation}
 F(\Bx)= \sum_{i=1}^{m} (r_i - \|{\Bx} - {\Ba}_i\|)^2 \nonumber
\end{equation}
\linespread{1}\selectfont
its Hessian for points $\Bx$ that are not coincided with $\Ba_i$ for $1 \leq i \leq m$, is given by
\begin{equation}
\begin{aligned}
\nonumber
\bigtriangledown ^2 F(\Bx)  = 2m\BI  + &2\sum^m_{i=1} \frac{r_i}{\|\Bx - \Ba_i\|^3} \cdot \\
\cdot &\left( \left(\Bx - \Ba_i\right)\left(\Bx - \Ba_i\right)^T - \|\Bx - \Ba_i\|^2 \BI \right)
\end{aligned}
\end{equation}
which is not always positive semidefinite. Hence $F(\Bx)$ is not convex.
\end{frame}


% A.1
\begin{frame}
\frametitle{Source Localization From Range Measurements}
{\large \textit{Weighted Squared Range Least Squares Formulation}} \\
\normalsize
\begin{itemize}
\item
Following [BSL2008], we convert \eqref{WSR} into a GTRS as
\setcounter{equation}{0} 
\setcounter{abc}{1}
\begin{eqnarray} \label{1} %eq 6 a,b
\Min_{{\By} \in R^{n+1}} \|\BA_w\By-\Bb_w\|^2 \qquad\\
\stepcounter{abc} 
\setcounter{equation}{1} 
\mbox{subject to: \ }
\By^T\BD\By + 2\Bf^T\By = 0
\end{eqnarray}
where $\By = [\Bx^T \ \alpha]^T$, $\alpha = \|\Bx\|$, $\BA_w = \BGA\BA$ and $\Bb_w = \BGA\Bb$ with fixed $\BGA=\mbox{diag}\left(\sqrt{w_1},\ldots,\sqrt{w_m}\right)$, and

\begin{equation} \label{2}
\setcounter{equation}{2}
\setcounter{abc}{0}
\BA=\left(\begin{array}{cc}
    -2\Ba_1^T & 1 \\
    \vdots  & \vdots \\
    -2\Ba_m^T & 1
    \end{array} \right),
\Bb=\left(\begin{array}{c}
    r_1^T-\|\Ba_1\|^T \\
    \vdots \\
    r_m^T-\|\Ba_m\|^T
    \end{array} \right)
\end{equation}
\begin{equation} \label{3}
\BD=\left(\begin{array}{cc}
    \BI\!_{n\times n} & \BO_{n\times 1} \\
    \BO_{1\times n} & 0
    \end{array} \right),
\Bf=\left(\begin{array}{c}\BO \\ -0.5 \end{array} \right)
\end{equation}
%For a \textit{fixed} set of weights $\{w_i, i = 1, …, m\}$, the global solution of (6) can be solved by the method developed in and we denote the solution by  $\mathbf{S}(\BA_w, \Bb_w, \BD, \Bf)$.
\end{itemize}
\end{frame}

%%------------------------------------------------
%\begin{frame}
%\frametitle{Source Localization From Range Measurements}
%{\large \textit{An Iterative Re-Weighting Strategy}} \\
%\normalsize
% More importantly, when the iterates produced by solving (\ref{eq:12}) (namely $\Bx_k$ for $k = 1, 2,\ldots$) converge, in the $k$th iteration with $k$ sufficiently large, the objective function of (\ref{eq:11}) in a small vicinity of its solution $\Bx_k$ is approximately equal to
%\begin{equation} \label{sr:w}
%\nonumber
%\begin{aligned}
%&\sum_{i=1}^m w_i^{(k)}\left(\|\Bx-\Ba_i\|^2-r_i^2\right)^2 \\ % \approx \sum_{i=1}^m w_i^{(k)}\left(\|\Bx_k-\Ba_i\|^2-r_i^2\right)^2 \\
%%&=\sum_{i=1}^m w_i^{(k)}\left(\|\Bx_k-\Ba_i\|+r_i\right)^2\left(\|\Bx_k-\Ba_i\|-r_i\right)^2  \\
%& \approx \sum_{i=1}^m w_i^{(k)}\left(\|\Bx_{k-1}-\Ba_i\|+r_i\right)^2\left(\|\Bx_k-\Ba_i\|-r_i\right)^2 
%%&=\sum_{i=1}^m \left(\|\Bx_k-\Ba_i\|-r_i\right)^2 
% \approx \sum_{i=1}^m \left(\|\Bx-\Ba_i\|-r_i\right)^2\\
%\\
%\end{aligned}
%\end{equation}
%\end{frame}
%%------------------------------------------------

% A.2
\begin{frame}
%\frametitle{An Iterative Re-Weighting Strategy}
\frametitle{Source Localization From Range Measurements}
{\large \textit{The Algorithm}} \\
\normalsize
% The localization method can be outlined as follows.
\begin{enumerate}
\item %1
\underline{Input data}: Sensor locations $\{\Ba_i, i=1,\ldots,m\}$, range measurements $\{r_i, i=1,\ldots,m\}$, maximum number of iterations $k_{max}$ and convergence tolerance $\zeta$.
\item %2
Generate data set $\BA,\Bb, \BD, \Bf$ using (2) and \eqref{3}. Set $k=1, w_i^{(1)}=1$ for $i=1,\ldots,m$.
\item %3 
Set $\BGA_k=\mbox{diag}\left(\sqrt{w_1^{(k)}},\ldots,\sqrt{w_m^{(k)}}\right)$, $\BA_w=\BGA_k\BA$ and $\Bb_w=\BGA_k\Bb$.
\item%4
Solve the WSR-LS problem \eqref{IRWSR}  via (1) to obtain its global solution $\Bx_k$. % = $ \mathbf{S}(\BA_w, \Bb_w, \BD, \Bf)$.
\item %5
If $k=k_{max}$ or $\|\Bx_k-\Bx_{k-1}\|<\zeta$, terminate and \underline{output $\Bx_k$} as the solution; otherwise, set $k=k+1$, update weights $\{w_i^{(k)}, i=1,\ldots,m\}$ and repeat from Step 3).
\end{enumerate}
\end{frame}
%%------------------------------------------------

%\begin{frame}
%\frametitle{The Algorithm}
%It is evident that the complexity of the algorithm is practically equal to the complexity of the WSR-LS solver involved in Step 4 times the number of iterations, $k$. The algorithm converges with a small number of iterations, typically a $k \leq 6$ suffices. Convergence is guaranteed due to the nature of the WSR-LS solver which produces a global solution that does not depend on the initial point, and convexity of the WSR-LS problem (6). For simplicity, we shall call the solutions obtained from Algorithm 1 IRWSR-LS solutions.
%\end{frame}

%------------------------------------------------

% A.3
\begin{frame}
\frametitle{Source Localization From Range-Difference Measurements} %3.2
{\large \textit{Weighted Squared Range-Difference Least Squares Formulation}} \\~\\
\normalsize
\begin{itemize}
\item 
By introducing new variable $\By=[\Bx^T \ \|\Bx\|]^T$ and noticing nonnegativity of the component $y_{n+1}$ problem \eqref{WSRD} is converted to
\begin{eqnarray} \label{4}
\setcounter{abc}{0} \stepcounter{abc}
\Min_{\By \in R^{n+1}} \|\BB_w\By - \Bg_w\| \\
\stepcounter{abc} \setcounter{equation}{4}
\mbox{subject to: } \By^T\BC\By = 0 \\
\stepcounter{abc} \setcounter{equation}{4}
y_{n+1}\geq 0
\end{eqnarray}
\item
where $\BB_w=\BGA\BB$, $\Bg_w=\BGA\Bg$ , $\BGA=\mbox{diag}\{\sqrt{w_1},\ldots,\sqrt{w_m}\}$,  $\Bg=[g_1 \ldots g_m]^T$ and
\begin{equation} \label{5}
\setcounter{abc}{0}
%\label{eq:20} %29
\BB = \left(\begin{array}{cc}
    -2\Ba_1^T & -2d_1 \\
    \vdots & \vdots \\
    -2\Ba_m^T & -2d_m
    \end{array}\right),
\BC =  \left(\begin{array}{cc}
    \BI_n & \BO_{n \times 1} \\
    \BO_{1 \times n} & -1
    \end{array}\right)
\end{equation}

%On comparing (23) with (19), it follows immediately that the global solver for problem (19) characterized by data set $\{\BB, \Bg, \BC\}$ can also be used for solving problem (23) by applying it to data set $\{\BB_w, \Bg_w, \BC\}$.

\end{itemize}
\end{frame}
%------------------------------------------------
%\begin{frame}
%\frametitle{SRD-LS Formulation}
%Reference [12] presents a rigorous argument which shows that the optimal solution of (\ref{eq:19}) either assumes the form of  $\tilde{\By}(\lambda)=\left(\BB_w^T\BB_w+\lambda\BC\right)^{-1}\BB^T\Bg$
% where $\lambda$ solves
% \begin{equation}\label{eq:21}
% \tilde{\By}(\lambda)^T\BC\tilde{\By}(\lambda)=0
% \end{equation}
% and makes $\BB^T\BB+\lambda\BC$ positive definite, or is the vector among $\{\BO,$ $\tilde{\By}(\lambda_1),\ldots,\tilde{\By}(\lambda_p)\}$ that gives the smallest objective function in (\ref{eq:19}a), where $\{\lambda_i, i = 1,\ldots,p\}$ are all roots of (\ref{eq:21}) such that the $(n+1)$'th component of $\tilde{\By}(\lambda_i)$ is nonnegative and $\BB^T\BB+\lambda\BC$ has exactly one negative and $n$ positive eigenvalues. We shall refer the global solution of (\ref{eq:19}) to as the SRD-LS solution.
%\end{frame}

%------------------------------------------------
% A.4 
\begin{frame}
%\frametitle{An Iterative Re-Weighting Strategy}
\frametitle{Source Localization From Range Difference Measurements}
{\large \textit{The Algorithm}} \\~\\
\normalsize
% The localization method can be outlined as follows.
\begin{enumerate}
\item %1 
\underline{Input data}: Sensor locations $\{\Ba_i, i=0, 1,\ldots,m\}$ with $\Ba_0=\BO$, range-difference measurements $\{d_i, i = 1,\ldots,m\}$, maximum number of iterations $k_{max}$ and convergence tolerance $\xi$.
\item %2
Generate data set $\{\BB, \Bg, \BC\}$ using (5). Set $k=1$, $w_i^{(1)}=1$ for $i=1,\ldots,m.$
\item %3
Set $\BGA_k=\mbox{diag}\left(\sqrt{w_1^{(k)}},\ldots,\sqrt{w_m^{(k)}}\right)$, $\BB_w=\BGA_k\BB$ and $\Bg_w=\BGA_k\Bg$.
\item %4
Solve WSRD-LS problem \eqref{4} to obtain its global solution $\Bx_k$.
\item %5
If $k=k_{max}$ or $\|\Bx_k-\Bx_{k-1}\|<\xi$, terminate and \underline{output $\Bx_k$} as the solution; otherwise, set $k=k+1$, update weights $\{w_i^{(k)}, i=1,\ldots,m\}$ and repeat from Step 3).
%We shall call the solutions obtained from Algorithm 2 IRWSRD-LS solutions.
\end{enumerate}
\end{frame}

%%%%%%%%%%%%%%%%%%%
%%%% PCCCP
%%%%%%%%%%%%%%%%%%%

%%------------------------------------------------
% A.2
\begin{frame} 
\frametitle{PCCP - Problem Reformulation}
We express the objective in (R) as $F(\Bx) = f(\Bx) –- g(\Bx)$ with 
 \begin{equation}  \nonumber 
f(\Bx) =  
m \Bx^T\Bx - 2 \Bx^T\sum^m_{i=1} \Ba_i \ \ \mbox{and } \ g(\Bx) = 2 \sum^m_{i = 1} r_i \|\Bx - \Ba_i\| 
\end{equation}
  
Then, we replace $\bigtriangledown g(\Bx_k)$ by a subgradient of $g(\Bx)$ at $\Bx_k$:
\begin{equation} 
\nonumber
\partial g{(\Bx_k)}  = 2 \sum^m_{i = 1} r_i \partial \|\Bx_k - \Ba_i\|, 
\end{equation}
where 
\begin{equation}
\nonumber
\partial \|\Bx_k - \Ba_i\|  = \left\{
	\begin{aligned}
	& \frac{\Bx_k - \Ba_i}{\|\Bx_k - \Ba_i\|}, &\mbox{if } \Bx_k \neq \Ba_i \\
	& \BO, &\mbox{otherwise }
	\end{aligned}
\right.
\end{equation}
\end{frame}

\begin{frame} 
\frametitle{PCCP - Problem Reformulation}
\phantom{m}
Hence  $ \hat{g}(\Bx,\Bx_k) $ can be formed as: 
\begin{equation} 
\nonumber
\begin{aligned}
  \hat{g}(\Bx,\Bx_k)   & =   g(\Bx_k) +  \bigtriangledown g(\Bx_k)^T(\Bx - \Bx_k) \\
  & =   2 \sum^m_{i=1} r_i \|\Bx_k - \Ba_i\|   +  2 \left( \Bx - \Bx_k \right)^{T} \sum^m_{i=1} r_i \partial \|\Bx_k - \Ba_i\| \\
& = 2\Bx^T \sum^m_{i=1} r_i \partial \|\Bx_k - \Ba_i \| + c
\end{aligned}
\end{equation} 
where $c$ is a constant given  by
\begin{equation} 
\nonumber
 c = - 2 \sum_{i = 1}^m r_i \Ba_i^T \partial \|\Bx_k - \Ba_i\|.
\end{equation} 
\end{frame}

%%------------------------------------------------
% A.3
\begin{frame}
\frametitle{PCCP-based LS Algorithm for Source Localization}
\phantom{m}
%{\large \textit{PCCP-based LS Algorithm for Source Localization}}
%\\~\\
\textbf{Step 1:} Input sensor locations $\{\Ba_i, i = 1,\ldots,m\}$, range measurements $\{r_i, i = 1, \ldots, m\}$, $\Bx_0, K_{max}, \tau_0, \tau_{max}, \mu > 0, \gamma, \sigma$, and set $k = 0$. 
\\~\\
\textbf{Step 2:} Form  $\Bv_k$ and solve PCCP. Denote the solution as  $(\Bs^*, \hat{\Bs}^*, \Bx^*)$. 
\\~\\
\textbf{Step 3:} Update  $\tau_{k+1} $ = min $(\mu\tau_k, \tau_{max})$, set $k = k+1$. 
\\~\\
\textbf{Step 4:} If $k = K_{max}$, terminate and output $\Bx^*$ as the solution; otherwise, set $\Bx_k = \Bx^*$ and repeat from Step 2. 
\end{frame}

%%------------------------------------------------
% A.4
\begin{frame}
\frametitle{Sequential Convex Relaxation for Range-Difference Localization}
\phantom{m}
\textbf{Step 1:} Input data: \\
- sensor locations $\{\Ba_i, i=1,\ldots,m\}$, \\
- range-difference measurements $\{d_i, i=1,\ldots,m\}$, \\
- initial point $\Bx_0$, \\
- initial weight $\tau_0$ and upper limit of weight $\tau_{max}$, \\
- increment bound $\beta$ \\
-  convergence tolerance $\epsilon$. %\gamma, \sigma$, 
Set iteration count to $k = 0$. \\
Form $\BS, \BC$ and $\Bq$ as
\begin{equation}
\nonumber
\BS = \begin{bmatrix}
\symb{0}_{m \times 1} & -\symb{1}_{m \times 1} & -\symb{I}_m
\end{bmatrix},
\BC = \begin{bmatrix}
\symb{I}_{m+3} \\
-\symb{I}_{m+3}
\end{bmatrix},
\Bq = \beta\Be
\end{equation}
\end{frame}

% A.5
\begin{frame} [t,allowframebreaks]
\frametitle{Sequential Convex Relaxation for Range-Difference Localization}
\phantom{m}
%\\~\\
\textbf{Step 2:} Form $y_k$ and $\Bz_k$ as
%\begin{equation}
%\nonumber
$y_k =\|\Bx_k\|, \Bz_k = \begin{bmatrix}
\|\Bx_k - \Ba_1\| \\
\vdots \\
\|\Bx_k - \Ba_m\|
\end{bmatrix} $
%\end{equation}
\\ Form $\BA_k, \tilde{\Bd_k}, \Bb_k$, $\BC_k $  and solve
\begin{eqnarray} 
\nonumber
\Min_{\Bdelta}& &f\left(\tilde{\Bdelta}\right) + \tau_k\sum_{i=1}^{m+1} \left(u_i + v_i\right) +\tau_k w\\
\nonumber
\mbox{subject to}& &\BA_k\tilde{\Bdelta} - \Bb_k = \Bu - \Bv \\
\nonumber
& &\BC\tilde{\Bdelta} - \Bq \leq w\Be \\
\nonumber
& & \Bu \geq \symb{0}, \Bv \geq \symb{0}, w \geq 0
\end{eqnarray}
\noindent
Denote the solution as $\tilde{\Bdelta}_k = (\Bdelta_x^*, \delta_y^*, \Bdelta_z^*)$.  
\\~\\
\textbf{Step 3:} Update  $\tau_{k+1} $ = min $(1.5\tau_k, \tau_{max})$, set $k = k+1$. Update $\tilde{\Bx}^{*}$ to
\begin{eqnarray} 
\nonumber
\Bx^{*} = \Bx^k + \Bdelta_x^* \\
\nonumber
y^{*} = y^k + \delta_y^* \\
\nonumber
\Bz^{*} = \Bz^k + \Bdelta_z^*
\end{eqnarray}
\\~\\
\textbf{Step 4:} If $\|\tilde{\Bdelta}_k\| \leq \epsilon$, terminate and output $\Bx^*$ as the solution; otherwise, set $\tilde{\Bx}_{k} = \Bx^{*}$  and repeat from Step 2. 
\end{frame}


%%------------------------------------------------
% A.6
\begin{frame} [t,allowframebreaks]
\frametitle{Sequential Convex Relaxation for Range-Based Localization}
\phantom{m}
\textbf{Step 1:} Input data: \\
- sensor locations $\{\Ba_i,  i=1,\ldots,m\}$, \\
- range measurements $\{r_i, i=1,\ldots,m\}$, \\
- initial point $\Bx_0$, initial relaxation parameter $\gamma_0$, \\
- the number of iterations to be executed $K_{max}$. %\gamma, \sigma$, 
\\ Set iteration count to $k = 0$.
\\~\\
\phantom{m}
\phantom{m}
\textbf{Step 2:} Solve
\begin{eqnarray} 
\nonumber
\Min_{\Bx, \Bz}& &\sum^m_i \left( z_i - r_i \right)^2 \\
\nonumber
\mbox{subject to:}& &\|\Bx - \Ba_i\|  \leq  (1+ \gamma)z_i  
\end{eqnarray}
\begin{equation}
\nonumber
\qquad \quad   -\|\Bx_k - \Ba_i\| - \partial\|\Bx_k -\Ba_i\|^T(\Bx - \Ba_i)  \leq  -(1 - \gamma)z_i, \ i = 1, 2, ..., m
\end{equation}
%\phantom{m}
\noindent
Denote the solution as $\tilde{\Bx}_k = (\Bx^*, \Bz^*)$. 
\\~\\
\textbf{Step 3:} Update  $\gamma_{k+1} = f(\gamma_k)$ linearly or quadratically. Set $k = k+1$. 
\\~\\
\textbf{Step 4:} If $k = K_{max}$, terminate and output $\Bx^*$ as the solution; otherwise, set $\Bx_{k} = \Bx^{*}$  and repeat from Step 2. 
\end{frame}

%------------------------------------------------

% A.5

\begin{frame} %[t,allowframebreaks]
\frametitle{References}
\phantom{m}
%\printbibliography
%\end{frame}
%

\begin{thebibliography}{99} 

%\bibitem{1}
%%[SA1987] 
%J. O. Smith and J. S. Abel, ``Closed-form least-squares source location
%estimation from range-difference measurements,'' {\em IEEE Trans. Acoust.,
%Speech Signal Process.}, vol. 12, pp. 1661--1669, Dec. 1987.
%
%\bibitem{2}
%%[SR1987] 
%H. Schau and A. Robinson, ``Passive source localization employing intersecting spherical surfaces from time-of-arrival differences,'' {\em IEEE Trans. Acoust., Speech Signal Process.}, vol. ASSP--35, pp. 1223--1225, Aug. 1987.
%
%\bibitem{3}
%%[YHR1998] 
%K. Yao, R. Hudson, C. Reed, D. Chen, and F. Lorenzelli, ``Blind beamforming on a randomly distributed sensor array system,'' {\em IEEE J. Select. Areas Commun.}, vol. 16, pp. 1555-1567, Oct. 1998.
%
%\bibitem{4}
%%[Sp2001]  
%M. A. Sprito, ``On the accuracy of cellular mobile station location estimation,'' {\em IEEE Trans. Veh. Technol.}, vol. 50, pp. 674-685, May 2001.
%
%\bibitem{5}
%%[HBE2002] 
%Y. Huang, J. Benesty, G. W. Elko, and R. M. Mersereau, ``Realtime passive source localization: A practical linear correction least-squares approach,'' {\em IEEE Trans. Speech Audio Process.}, vol. 9, no. 8, pp. 943-956, Nov. 2002.
%
%\bibitem{6}
%%[CSM2004]
%K. W. Cheung, H. C. So, W. K. Ma, and Y. T. Chan, ``Least squares algorithms for time-of-arrival-based mobile location,'' {\em IEEE Trans. Signal Process.}, vol. 52, no. 4, pp. 1121--1228, Apr. 2004.
%
%\bibitem{7}
%%[LH2004] 
%D. Li and H. Hu, ``Least square solutions of energy based acoustic source localization problems,'' in {\em Proc. ICPPW}, 2004.
%
%\bibitem{8}
%%[CMS2004] 
%K.W. Cheung, W.K. Ma, and H.C. So, ``Accurate approximation algorithm for TOA-based maximum-likelihood mobile location using semidefinite programming,'' in {\em Proc. ICASSP}, 2004, vol. 2, pp. 145--148.
%
%\bibitem{9}
%%[STK2005] 
%A. H. Sayed, A. Tarighat, and N. Khajehnouri, ``Network-based wireless location,'' {\em IEEE Signal Process. Mag.}, vol. 22, no. 4, pp. 24--40, July 2005.
%
%\bibitem{10}
%%[CHC2006] 
%Y. T. Chan, H. Y. C. Hang, and P. C. Ching, ``Exact and approximate maximum likelihood localization algorithms,'' {\em IEEE Trans. Veh. Technol.}, vol. 55, no. 1, pp. 10--16, Jan. 2006.
%
\bibitem{11}
[SL2006] 
P. Stoica and J. Li, ``Source localization from range-difference measurements,'' {\em IEEE Signal Process. Mag.}, vol. 23, pp. 63--65,69, Nov. 2006.

\bibitem{12}
[BSL2008] 
A. Beck, P. Stoica and J. Li,  ``Exact and approximate solutions of source localization problems,'' {\em IEEE Trans. Signal Processing}, vol. 56, no. 5, pp. 1770--1777, May 2008.

\bibitem{13}
[GeoLoc] 
C. Gentile, N. Alsindi, R. Raulefs, and C. Teolis,   {\em Geolocation
Techniques: Principles and Applications}, vol. 56, no. 5, pp. 1770--1777, May 2008.
%
%\bibitem{13}
%%[VB1996] 
%L. Vandenberghe and S. Boyd, ``Semidefinite programming,'' {\em SIAM Rev.}, vol. 38, no. 1, pp. 40--95, Mar. 1996.
%
%\bibitem{14}
%%[AL2007] 
%A. Antoniou and W.-S. Lu, {\em Practical Optimization: Algorithms and Engineering Applications}, Springer, 2007.
%
%\bibitem{15}
%%[Mo1993] 
%J.J. More, ``Generalizations of the trust region subproblem,'' {\em Optim. Methods Softw.}, vol. 2, pp. 189--209, 1993.
%
%\bibitem{16}
%%[FW2004] 
%C. Fortin and H. Wolkowicz, ``The trust region subproblem and semidefinite programming,'' {\em Optim. Methods Softw.}, vol. 19, no.1, pp. 41--67, 2004.

\end{thebibliography}


%\begin{frame}
%\frametitle{References}
%\footnotesize{
%\begin{thebibliography}{99} % Beamer does not support BibTeX so references must be inserted manually as below
%\bibitem[Smith, 2012]{p1} John Smith (2012)
%\newblock Title of the publication
%\newblock \emph{Journal Name} 12(3), 45 -- 678.
%\end{thebibliography}
%}
\end{frame}

%%--------------------------------------------------
% EXAMPLES %--------------------------------------------------
%%\frametitle{Blocks of Highlighted Text}
%\begin{block}{Block 1}
%Lorem
%\end{block}
%
%\begin{block}{Block 2}
%Pellentesque sed tellus purus.
%\end{block}
%
%\begin{block}{Block 3}
%Suspendisse tincidunt sagittis gravida.
%\end{block}
%\end{frame}

%------------------------------------------------

%\begin{frame}
%\frametitle{Multiple Columns}
%\begin{columns}[c] % The "c" option specifies centered vertical alignment while the "t" option is used for top vertical alignment
%
%\column{.45\textwidth} % Left column and width
%\textbf{Heading}
%\begin{enumerate}
%\item Statement
%\item Explanation
%\item Example
%\end{enumerate}
%
%\column{.5\textwidth} % Right column and width
%Lorem ipsum dolor sit amet, consectetur adipiscing elit. Integer lectus nisl, ultricies in feugiat rutrum, porttitor sit amet augue. Aliquam ut tortor mauris. Sed volutpat ante purus, quis accumsan dolor.
%
%\end{columns}
%\end{frame}



%----------------------------------------------------------------------------------------

\end{document} 