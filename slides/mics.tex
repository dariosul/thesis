%%%%%%%%%%%%%
%%%b IRW
%%%%%%%%%%%%%

% A.1
\begin{frame}
\frametitle{Nonconvexity of the R-LS objective}
\phantom{m}
\linespread{0.1} \selectfont
Given the objective
\begin{equation}
 F(\Bx)= \sum_{i=1}^{m} (r_i - \|{\Bx} - {\Ba}_i\|)^2 \nonumber
\end{equation}
\linespread{1}\selectfont
its Hessian for points $\Bx$ that are not coincided with $\Ba_i$ for $1 \leq i \leq m$, is given by
\begin{equation}
\begin{aligned}
\nonumber
\bigtriangledown ^2 F(\Bx)  = 2m\BI  + &2\sum^m_{i=1} \frac{r_i}{\|\Bx - \Ba_i\|^3} \cdot \\
\cdot &\left( \left(\Bx - \Ba_i\right)\left(\Bx - \Ba_i\right)^T - \|\Bx - \Ba_i\|^2 \BI \right)
\end{aligned}
\end{equation}
which is not always positive semidefinite. Hence $F(\Bx)$ is not convex.
\end{frame}


% A.1
\begin{frame}
\frametitle{Source Localization From Range Measurements}
{\large \textit{Weighted Squared Range Least Squares Formulation}} \\
\normalsize
\begin{itemize}
\item
Following [BSL2008], we convert \eqref{WSR} into a GTRS as
\setcounter{equation}{0} 
\setcounter{abc}{1}
\begin{eqnarray} \label{1} %eq 6 a,b
\Min_{{\By} \in R^{n+1}} \|\BA_w\By-\Bb_w\|^2 \qquad\\
\stepcounter{abc} 
\setcounter{equation}{1} 
\mbox{subject to: \ }
\By^T\BD\By + 2\Bf^T\By = 0
\end{eqnarray}
where $\By = [\Bx^T \ \alpha]^T$, $\alpha = \|\Bx\|$, $\BA_w = \BGA\BA$ and $\Bb_w = \BGA\Bb$ with fixed $\BGA=\mbox{diag}\left(\sqrt{w_1},\ldots,\sqrt{w_m}\right)$, and

\begin{equation} \label{2}
\setcounter{equation}{2}
\setcounter{abc}{0}
\BA=\left(\begin{array}{cc}
    -2\Ba_1^T & 1 \\
    \vdots  & \vdots \\
    -2\Ba_m^T & 1
    \end{array} \right),
\Bb=\left(\begin{array}{c}
    r_1^T-\|\Ba_1\|^T \\
    \vdots \\
    r_m^T-\|\Ba_m\|^T
    \end{array} \right)
\end{equation}
\begin{equation} \label{3}
\BD=\left(\begin{array}{cc}
    \BI\!_{n\times n} & \BO_{n\times 1} \\
    \BO_{1\times n} & 0
    \end{array} \right),
\Bf=\left(\begin{array}{c}\BO \\ -0.5 \end{array} \right)
\end{equation}
%For a \textit{fixed} set of weights $\{w_i, i = 1, …, m\}$, the global solution of (6) can be solved by the method developed in and we denote the solution by  $\mathbf{S}(\BA_w, \Bb_w, \BD, \Bf)$.
\end{itemize}
\end{frame}

%%------------------------------------------------
%\begin{frame}
%\frametitle{Source Localization From Range Measurements}
%{\large \textit{An Iterative Re-Weighting Strategy}} \\
%\normalsize
% More importantly, when the iterates produced by solving (\ref{eq:12}) (namely $\Bx_k$ for $k = 1, 2,\ldots$) converge, in the $k$th iteration with $k$ sufficiently large, the objective function of (\ref{eq:11}) in a small vicinity of its solution $\Bx_k$ is approximately equal to
%\begin{equation} \label{sr:w}
%\nonumber
%\begin{aligned}
%&\sum_{i=1}^m w_i^{(k)}\left(\|\Bx-\Ba_i\|^2-r_i^2\right)^2 \\ % \approx \sum_{i=1}^m w_i^{(k)}\left(\|\Bx_k-\Ba_i\|^2-r_i^2\right)^2 \\
%%&=\sum_{i=1}^m w_i^{(k)}\left(\|\Bx_k-\Ba_i\|+r_i\right)^2\left(\|\Bx_k-\Ba_i\|-r_i\right)^2  \\
%& \approx \sum_{i=1}^m w_i^{(k)}\left(\|\Bx_{k-1}-\Ba_i\|+r_i\right)^2\left(\|\Bx_k-\Ba_i\|-r_i\right)^2 
%%&=\sum_{i=1}^m \left(\|\Bx_k-\Ba_i\|-r_i\right)^2 
% \approx \sum_{i=1}^m \left(\|\Bx-\Ba_i\|-r_i\right)^2\\
%\\
%\end{aligned}
%\end{equation}
%\end{frame}
%%------------------------------------------------

% A.2
\begin{frame}
%\frametitle{An Iterative Re-Weighting Strategy}
\frametitle{Source Localization From Range Measurements}
{\large \textit{The Algorithm}} \\
\normalsize
% The localization method can be outlined as follows.
\begin{enumerate}
\item %1
\underline{Input data}: Sensor locations $\{\Ba_i, i=1,\ldots,m\}$, range measurements $\{r_i, i=1,\ldots,m\}$, maximum number of iterations $k_{max}$ and convergence tolerance $\zeta$.
\item %2
Generate data set $\BA,\Bb, \BD, \Bf$ using (2) and \eqref{3}. Set $k=1, w_i^{(1)}=1$ for $i=1,\ldots,m$.
\item %3 
Set $\BGA_k=\mbox{diag}\left(\sqrt{w_1^{(k)}},\ldots,\sqrt{w_m^{(k)}}\right)$, $\BA_w=\BGA_k\BA$ and $\Bb_w=\BGA_k\Bb$.
\item%4
Solve the WSR-LS problem \eqref{IRWSR}  via (1) to obtain its global solution $\Bx_k$. % = $ \mathbf{S}(\BA_w, \Bb_w, \BD, \Bf)$.
\item %5
If $k=k_{max}$ or $\|\Bx_k-\Bx_{k-1}\|<\zeta$, terminate and \underline{output $\Bx_k$} as the solution; otherwise, set $k=k+1$, update weights $\{w_i^{(k)}, i=1,\ldots,m\}$ and repeat from Step 3).
\end{enumerate}
\end{frame}
%%------------------------------------------------

%\begin{frame}
%\frametitle{The Algorithm}
%It is evident that the complexity of the algorithm is practically equal to the complexity of the WSR-LS solver involved in Step 4 times the number of iterations, $k$. The algorithm converges with a small number of iterations, typically a $k \leq 6$ suffices. Convergence is guaranteed due to the nature of the WSR-LS solver which produces a global solution that does not depend on the initial point, and convexity of the WSR-LS problem (6). For simplicity, we shall call the solutions obtained from Algorithm 1 IRWSR-LS solutions.
%\end{frame}

%------------------------------------------------

% A.3
\begin{frame}
\frametitle{Source Localization From Range-Difference Measurements} %3.2
{\large \textit{Weighted Squared Range-Difference Least Squares Formulation}} \\~\\
\normalsize
\begin{itemize}
\item 
By introducing new variable $\By=[\Bx^T \ \|\Bx\|]^T$ and noticing nonnegativity of the component $y_{n+1}$ problem \eqref{WSRD} is converted to
\begin{eqnarray} \label{4}
\setcounter{abc}{0} \stepcounter{abc}
\Min_{\By \in R^{n+1}} \|\BB_w\By - \Bg_w\| \\
\stepcounter{abc} \setcounter{equation}{4}
\mbox{subject to: } \By^T\BC\By = 0 \\
\stepcounter{abc} \setcounter{equation}{4}
y_{n+1}\geq 0
\end{eqnarray}
\item
where $\BB_w=\BGA\BB$, $\Bg_w=\BGA\Bg$ , $\BGA=\mbox{diag}\{\sqrt{w_1},\ldots,\sqrt{w_m}\}$,  $\Bg=[g_1 \ldots g_m]^T$ and
\begin{equation} \label{5}
\setcounter{abc}{0}
%\label{eq:20} %29
\BB = \left(\begin{array}{cc}
    -2\Ba_1^T & -2d_1 \\
    \vdots & \vdots \\
    -2\Ba_m^T & -2d_m
    \end{array}\right),
\BC =  \left(\begin{array}{cc}
    \BI_n & \BO_{n \times 1} \\
    \BO_{1 \times n} & -1
    \end{array}\right)
\end{equation}

%On comparing (23) with (19), it follows immediately that the global solver for problem (19) characterized by data set $\{\BB, \Bg, \BC\}$ can also be used for solving problem (23) by applying it to data set $\{\BB_w, \Bg_w, \BC\}$.

\end{itemize}
\end{frame}
%------------------------------------------------
%\begin{frame}
%\frametitle{SRD-LS Formulation}
%Reference [12] presents a rigorous argument which shows that the optimal solution of (\ref{eq:19}) either assumes the form of  $\tilde{\By}(\lambda)=\left(\BB_w^T\BB_w+\lambda\BC\right)^{-1}\BB^T\Bg$
% where $\lambda$ solves
% \begin{equation}\label{eq:21}
% \tilde{\By}(\lambda)^T\BC\tilde{\By}(\lambda)=0
% \end{equation}
% and makes $\BB^T\BB+\lambda\BC$ positive definite, or is the vector among $\{\BO,$ $\tilde{\By}(\lambda_1),\ldots,\tilde{\By}(\lambda_p)\}$ that gives the smallest objective function in (\ref{eq:19}a), where $\{\lambda_i, i = 1,\ldots,p\}$ are all roots of (\ref{eq:21}) such that the $(n+1)$'th component of $\tilde{\By}(\lambda_i)$ is nonnegative and $\BB^T\BB+\lambda\BC$ has exactly one negative and $n$ positive eigenvalues. We shall refer the global solution of (\ref{eq:19}) to as the SRD-LS solution.
%\end{frame}

%------------------------------------------------
% A.4 
\begin{frame}
%\frametitle{An Iterative Re-Weighting Strategy}
\frametitle{Source Localization From Range Difference Measurements}
{\large \textit{The Algorithm}} \\~\\
\normalsize
% The localization method can be outlined as follows.
\begin{enumerate}
\item %1 
\underline{Input data}: Sensor locations $\{\Ba_i, i=0, 1,\ldots,m\}$ with $\Ba_0=\BO$, range-difference measurements $\{d_i, i = 1,\ldots,m\}$, maximum number of iterations $k_{max}$ and convergence tolerance $\xi$.
\item %2
Generate data set $\{\BB, \Bg, \BC\}$ using (5). Set $k=1$, $w_i^{(1)}=1$ for $i=1,\ldots,m.$
\item %3
Set $\BGA_k=\mbox{diag}\left(\sqrt{w_1^{(k)}},\ldots,\sqrt{w_m^{(k)}}\right)$, $\BB_w=\BGA_k\BB$ and $\Bg_w=\BGA_k\Bg$.
\item %4
Solve WSRD-LS problem \eqref{4} to obtain its global solution $\Bx_k$.
\item %5
If $k=k_{max}$ or $\|\Bx_k-\Bx_{k-1}\|<\xi$, terminate and \underline{output $\Bx_k$} as the solution; otherwise, set $k=k+1$, update weights $\{w_i^{(k)}, i=1,\ldots,m\}$ and repeat from Step 3).
%We shall call the solutions obtained from Algorithm 2 IRWSRD-LS solutions.
\end{enumerate}
\end{frame}

%%%%%%%%%%%%%%%%%%%
%%%% PCCCP
%%%%%%%%%%%%%%%%%%%

%%------------------------------------------------
% A.2
\begin{frame} 
\frametitle{PCCP - Problem Reformulation}
We express the objective in (R) as $F(\Bx) = f(\Bx) –- g(\Bx)$ with 
 \begin{equation}  \nonumber 
f(\Bx) =  
m \Bx^T\Bx - 2 \Bx^T\sum^m_{i=1} \Ba_i \ \ \mbox{and } \ g(\Bx) = 2 \sum^m_{i = 1} r_i \|\Bx - \Ba_i\| 
\end{equation}
  
Then, we replace $\bigtriangledown g(\Bx_k)$ by a subgradient of $g(\Bx)$ at $\Bx_k$:
\begin{equation} 
\nonumber
\partial g{(\Bx_k)}  = 2 \sum^m_{i = 1} r_i \partial \|\Bx_k - \Ba_i\|, 
\end{equation}
where 
\begin{equation}
\nonumber
\partial \|\Bx_k - \Ba_i\|  = \left\{
	\begin{aligned}
	& \frac{\Bx_k - \Ba_i}{\|\Bx_k - \Ba_i\|}, &\mbox{if } \Bx_k \neq \Ba_i \\
	& \BO, &\mbox{otherwise }
	\end{aligned}
\right.
\end{equation}
\end{frame}

\begin{frame} 
\frametitle{PCCP - Problem Reformulation}
\phantom{m}
Hence  $ \hat{g}(\Bx,\Bx_k) $ can be formed as: 
\begin{equation} 
\nonumber
\begin{aligned}
  \hat{g}(\Bx,\Bx_k)   & =   g(\Bx_k) +  \bigtriangledown g(\Bx_k)^T(\Bx - \Bx_k) \\
  & =   2 \sum^m_{i=1} r_i \|\Bx_k - \Ba_i\|   +  2 \left( \Bx - \Bx_k \right)^{T} \sum^m_{i=1} r_i \partial \|\Bx_k - \Ba_i\| \\
& = 2\Bx^T \sum^m_{i=1} r_i \partial \|\Bx_k - \Ba_i \| + c
\end{aligned}
\end{equation} 
where $c$ is a constant given  by
\begin{equation} 
\nonumber
 c = - 2 \sum_{i = 1}^m r_i \Ba_i^T \partial \|\Bx_k - \Ba_i\|.
\end{equation} 
\end{frame}

%%------------------------------------------------
% A.3
\begin{frame}
\frametitle{PCCP-based LS Algorithm for Source Localization}
\phantom{m}
%{\large \textit{PCCP-based LS Algorithm for Source Localization}}
%\\~\\
\textbf{Step 1:} Input sensor locations $\{\Ba_i, i = 1,\ldots,m\}$, range measurements $\{r_i, i = 1, \ldots, m\}$, $\Bx_0, K_{max}, \tau_0, \tau_{max}, \mu > 0, \gamma, \sigma$, and set $k = 0$. 
\\~\\
\textbf{Step 2:} Form  $\Bv_k$ and solve PCCP. Denote the solution as  $(\Bs^*, \hat{\Bs}^*, \Bx^*)$. 
\\~\\
\textbf{Step 3:} Update  $\tau_{k+1} $ = min $(\mu\tau_k, \tau_{max})$, set $k = k+1$. 
\\~\\
\textbf{Step 4:} If $k = K_{max}$, terminate and output $\Bx^*$ as the solution; otherwise, set $\Bx_k = \Bx^*$ and repeat from Step 2. 
\end{frame}


%------------------------------------------------

% A.5

\begin{frame} [t,allowframebreaks]
\frametitle{References}

%\printbibliography
%\end{frame}
%

\begin{thebibliography}{99} 

\bibitem{1}
%[SA1987] 
J. O. Smith and J. S. Abel, ``Closed-form least-squares source location
estimation from range-difference measurements,'' {\em IEEE Trans. Acoust.,
Speech Signal Process.}, vol. 12, pp. 1661--1669, Dec. 1987.

\bibitem{2}
%[SR1987] 
H. Schau and A. Robinson, ``Passive source localization employing intersecting spherical surfaces from time-of-arrival differences,'' {\em IEEE Trans. Acoust., Speech Signal Process.}, vol. ASSP--35, pp. 1223--1225, Aug. 1987.

\bibitem{3}
%[YHR1998] 
K. Yao, R. Hudson, C. Reed, D. Chen, and F. Lorenzelli, ``Blind beamforming on a randomly distributed sensor array system,'' {\em IEEE J. Select. Areas Commun.}, vol. 16, pp. 1555-1567, Oct. 1998.

\bibitem{4}
%[Sp2001]  
M. A. Sprito, ``On the accuracy of cellular mobile station location estimation,'' {\em IEEE Trans. Veh. Technol.}, vol. 50, pp. 674-685, May 2001.

\bibitem{5}
%[HBE2002] 
Y. Huang, J. Benesty, G. W. Elko, and R. M. Mersereau, ``Realtime passive source localization: A practical linear correction least-squares approach,'' {\em IEEE Trans. Speech Audio Process.}, vol. 9, no. 8, pp. 943-956, Nov. 2002.

\bibitem{6}
%[CSM2004]
K. W. Cheung, H. C. So, W. K. Ma, and Y. T. Chan, ``Least squares algorithms for time-of-arrival-based mobile location,'' {\em IEEE Trans. Signal Process.}, vol. 52, no. 4, pp. 1121--1228, Apr. 2004.

\bibitem{7}
%[LH2004] 
D. Li and H. Hu, ``Least square solutions of energy based acoustic source localization problems,'' in {\em Proc. ICPPW}, 2004.

\bibitem{8}
%[CMS2004] 
K.W. Cheung, W.K. Ma, and H.C. So, ``Accurate approximation algorithm for TOA-based maximum-likelihood mobile location using semidefinite programming,'' in {\em Proc. ICASSP}, 2004, vol. 2, pp. 145--148.

\bibitem{9}
%[STK2005] 
A. H. Sayed, A. Tarighat, and N. Khajehnouri, ``Network-based wireless location,'' {\em IEEE Signal Process. Mag.}, vol. 22, no. 4, pp. 24--40, July 2005.

\bibitem{10}
%[CHC2006] 
Y. T. Chan, H. Y. C. Hang, and P. C. Ching, ``Exact and approximate maximum likelihood localization algorithms,'' {\em IEEE Trans. Veh. Technol.}, vol. 55, no. 1, pp. 10--16, Jan. 2006.

\bibitem{11}
%[SL2006] 
P. Stoica and J. Li, ``Source localization from range-difference measurements,'' {\em IEEE Signal Process. Mag.}, vol. 23, pp. 63--65,69, Nov. 2006.

\bibitem{12}
%[BSL2008] 
A. Beck, P. Stoica and J. Li,  ``Exact and approximate solutions of source localization problems,'' {\em IEEE Trans. Signal Processing}, vol. 56, no. 5, pp. 1770--1777, May 2008.

\bibitem{13}
%[VB1996] 
L. Vandenberghe and S. Boyd, ``Semidefinite programming,'' {\em SIAM Rev.}, vol. 38, no. 1, pp. 40--95, Mar. 1996.

\bibitem{14}
%[AL2007] 
A. Antoniou and W.-S. Lu, {\em Practical Optimization: Algorithms and Engineering Applications}, Springer, 2007.

\bibitem{15}
%[Mo1993] 
J.J. More, ``Generalizations of the trust region subproblem,'' {\em Optim. Methods Softw.}, vol. 2, pp. 189--209, 1993.

\bibitem{16}
%[FW2004] 
C. Fortin and H. Wolkowicz, ``The trust region subproblem and semidefinite programming,'' {\em Optim. Methods Softw.}, vol. 19, no.1, pp. 41--67, 2004.

\end{thebibliography}


%\begin{frame}
%\frametitle{References}
%\footnotesize{
%\begin{thebibliography}{99} % Beamer does not support BibTeX so references must be inserted manually as below
%\bibitem[Smith, 2012]{p1} John Smith (2012)
%\newblock Title of the publication
%\newblock \emph{Journal Name} 12(3), 45 -- 678.
%\end{thebibliography}
%}
\end{frame}