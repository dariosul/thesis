\startappendix{Matlab Code Listings}
\label{chapter:app2}


\section{Iterative Re-Weighting LS  for Source Localization Using Range Measurements}
\subsection{IRWSR-LS Algorithm}

The followng m-files implement the iterative re-weighting range-based least squares (IRWSR-LS) algorithm developed in Sec. 2.1 of the thesis. 

\phantom{m}

\noindent
\textbf{Input parameters}

\noindent
$\mbox{\textbf{Am}} = [\Ba_1 \ \Ba_2 \ldots \Ba_m]$, matrix of sensor locations, where $\Ba_i$ is a column vector that stores the coordinates of the $i$th sensor.

\noindent
$\Br_n = [r_1 \ r_2 \ldots r_m]^T$, noisy range measurements, where $r_i$ is a measurement obtained from the $i$th sensor.

\noindent
\textbf{Tunable parameters}

\noindent
$epsi$, convergence tolerance for the bisection routine.

\noindent
$K$, number of iterations for re-weighting procedure.

\noindent
\textbf{Otput parameters}

\noindent
$\Bx_w$, estimated location of the source.

\noindent
\textbf{Calling sequence}

\noindent
Function $\mbox{sr\_ls\_irw.m}$ calls $\mbox{gtrs\_r.m}$ to find the solution of the GTRS problem (2.16) in Step 3 of the algorithm.

\phantom{m}

% Example: xw = sr_ls_irw(Am,rn,1e-4,4);
\begin{lstlisting}
function xw = sr_ls_irw(Am,rn,epsi)
[n,m] = size(Am);
A = [-2*Am' ones(m,1)];
b = zeros(m,1);
for i = 1:m,
    ai = Am(:,i);
    b(i) = rn(i)^2 - ai'*ai;
end
z = zeros(n,1);
D = [eye(n) z; z' 0];
f = [z; -0.5];
yk = gtrs_r(A,b,D,f,epsi);
xk = yk(1:n);
wk = zeros(m,1);
ep = 1e-3;
err=1;
Kmax=10; 
K=1;
while err >= 1e-9 &&  K <= Kmax 
    for j = 1:m,
        nj = abs(norm(xk-Am(:,j))+rn(j));
        if nj < ep,
            wk(j) = 1/ep;
        else
            wk(j) = 1/nj;
        end
    end
    Gk = diag(wk);
    Agk = Gk*A;
    bgk = Gk*b;
    yk = gtrs_r(Agk,bgk,D,f,epsi);
    err=norm(xk-yk(1:n));
    xk = yk(1:n);
    K=K+1;
end
xw = xk;
\end{lstlisting}

\begin{lstlisting}
function y = gtrs_r(A,b,D,f,epsi)
% Find the left end of the interval I in (2.16)
A1 = A'*A;
atb = A'*b;
[U1,D1] = eig(A1);
d1 = diag(D1);
d1i = 1./sqrt(d1);
Si = U1*diag(d1i);
q = -1/max(eig(Si'*D*Si));
bL = q + 1e-26;
bu0 = abs(bL);
% Get an upper bound of lambda
bU = bu0;
yt = inv(A1+bU*D)*(atb-bU*f);
phit = (D*yt + 2*f)'*yt;
while phit > 1e-3,
    bU = bU + bu0;
    yt = inv(A1+bU*D)*(atb-bU*f);
    phit = (D*yt + 2*f)'*yt;
end
% Perform bisectioning
L = bL;
U = bU;
dt = U - L;
while dt > epsi,
    t = 0.5*(L + U);
    dt = 0.5*dt;
    yt = inv(A1+t*D)*(atb-t*f);
    phit = (D*yt + 2*f)'*yt;
    if phit > 0,
       L = t;
    else
       U = t;
    end
end
t = 0.5*(L + U);
% the best value of lambda
% Solution of GTRS is obtained as:
y = inv(A1+t*D)*(atb-t*f);
\end{lstlisting}

\subsection{Hybrid IRWSR-LS}

The followng m-files implement the \textit{hybrid} iterative re-weighting range-based least squares (IRWSR-LS) algorithm developed in Sec. 2.1 of the thesis. 

\phantom{m}

\noindent
\textbf{Input parameters}

\noindent
$fname$, name of the function to calculate the value of the LS objective at a point.

\noindent
$gname$, name of the function to calculate the gradient of the LS objective at a point.

\noindent
$hname$, name of the function to calculate the Hessian of the LS objective at a point.

\noindent
$\mbox{\textbf{Am}} = [\Ba_1 \ \Ba_2 \ldots \Ba_m]$, matrix of sensor locations, where $\Ba_i$ is a column vector that stores the coordinates of the $i$th sensor.

\noindent
$\Br_n = [r_1 \ r_2 \ldots r_m]^T$, noisy range measurements, where $r_i$ is a measurement obtained from the $i$th sensor.

\noindent
$\Bx_0$, solution of the $\mbox{sr\_ls\_irw.m}$ as an initial point for the Newton algorithm.

\noindent
\textbf{Tunable parameters}

\noindent
$\mbox{dt}$, small positive constant to modify the Hessian $\BH(\Bx)$ of the objective in case it is not positive definite.

\noindent
$epsi$, convergence tolerance.

\noindent
\textbf{Otput parameters}

\noindent
$\Bx_s$, estimated location of the source.

\noindent
$k$, number of iterations it took the algorithm to  converge to the solution point with the convergence tolerance $epsi$.

\noindent
$f_s$, value of the LS objective at the estimated soluion point.

\noindent
\textbf{Calling sequence}

\noindent
Function $\mbox{newton.m}$ calls $\mbox{inex\_lsearch.m}$ to perform the line search and $\mbox{Hx\_R\_LS.m}$ to evaluate the Hessian of the LS objective at a point. Function $\mbox{inex\_lsearch.m}$ calls $\mbox{fx\_R\_LS.m}$ and $\mbox{gx\_R\_LS.m}$ to evaluate the objective and the gradient of the LS objective at a point respectively.  

\phantom{m}

\begin{lstlisting}
function [xs,fs,k] = newton(fname,gname,hname,Am, rn, x0,dt,epsi)
max_it = 1e+4;
k = 1;
[n,m] = size(Am);
xk = x0;
gk = feval(gname,Am, rn, xk);
[Hk, tauk] = feval(hname,Am, rn, xk);
[V,D] = eig(Hk);
di = diag(D)+tauk;
% % ensure that Hk is PD
dmin = min(di);
if dmin > 0
    Hki = V*diag(1./di)*V';
else
    ind = find(di<=0);
    di(ind) = dt;
    Hki = V*diag(1./di)*V';
end
dk = -Hki*gk;
ak = inex_lsearch(Am, rn, xk,dk,fname,gname);
adk = ak*dk;
er = norm(adk);
it = 0;
while (er >= epsi) && (it <= max_it)
    xk = xk + adk;
    gk = feval(gname,Am, rn, xk);
    [Hk, tauk] = feval(hname,Am, rn, xk);
    [V,D] = eig(Hk);
    di = diag(D)+tauk;
    dmin = min(di);
    dmin = min(di);
    if dmin > 0
        Hki = V*diag(1./di)*V';
    else
        for i=1:n
            di(i) = max(di(i), dt);
        end
        Hki = V*diag(1./di)*V';
    end
    dk = -Hki*gk;
    ak = inex_lsearch(Am, rn, xk,dk,fname,gname);
    adk = ak*dk;
    er = norm(adk);
    k = k + 1;
    it = it+1;
end
xs = xk + adk;
fs = feval(fname,Am, rn, xs);
\end{lstlisting}

\begin{lstlisting}
function z = inex_lsearch(Am, rn, xk,s,fname,gname)
k = 0;
m = 0;
tau = 0.1;
chi = 0.75;
rho = 0.1;
sigma = 0.1;
mhat = 400;
epsilon = 1e-10;
xk = xk(:);
s = s(:);
% compute f0 and g0
    f0 = feval(fname,Am, rn, xk);
    gk = feval(gname,Am, rn, xk);
  m = m+2;
  deltaf0 = f0;
% step 2 Initialize line search
  dk = s;
  aL = 0;
  aU = 1e99;
  fL = f0;
  dfL = gk'*dk;
  if abs(dfL) > epsilon,
    a0 = -2*deltaf0/dfL;
  else
    a0 = 1;
 end
 if ((a0 <= 1e-9)|(a0 > 1)),
    a0 = 1;
 end
%step 3
 while 1,
    deltak = a0*dk;
    f0 = feval(fname, Am, rn, xk+deltak);
    m = m + 1;
%step 4
    if ((f0 > (fL + rho*(a0 - aL)*dfL)) ...
    & (abs(fL - f0) > epsilon) & (m < mhat))
      if (a0 < aU)
        aU = a0;
      end
      % compute a0hat using equation
      a0hat = aL + ((a0 - aL)^2*dfL)/(2*(fL - f0 + (a0 - aL)*dfL));
      a0Lhat = aL + tau*(aU - aL);
      if (a0hat < a0Lhat)
        a0hat = a0Lhat;
      end
      a0Uhat = aU - tau*(aU - aL);
      if (a0hat > a0Uhat)
        a0hat = a0Uhat;
      end
      a0 = a0hat;
    else
      gtemp = feval(gname,Am, rn, xk+a0*dk);
      df0 = gtemp'*dk;
      m = m + 1;
      % step 6
      if (((df0 < sigma*dfL) & (abs(fL - f0) > epsilon) ...
       & (m < mhat) & (dfL ~= df0)))
        deltaa0 = (a0 - aL)*df0/(dfL - df0);
        if (deltaa0 <= 0)
          a0hat = 2*a0;
        else
          a0hat = a0 + deltaa0;
        end
        a0Uhat = a0 + chi*(aU - a0);
        if (a0hat > a0Uhat)
          a0hat = a0Uhat;
        end
        aL = a0;
        a0 = a0hat;
        fL = f0;
        dfL = df0;
      else
        break;
      end
    end
 end % while 1
 if a0 < 1e-5,
    z = 1e-5;
 else
    z = a0;
 end 
\end{lstlisting}


\begin{lstlisting}
  function f = fx_R_LS(Am, rn,x0)
[n,m] = size(Am);
f = 0;
for i=1:m
    f = f + (norm(x0-Am(:,i))-rn(i))^2;
end
f = f/2;

function g = gx_R_LS(Am, rn,x0)
[n,m] = size(Am);
g = zeros(n,1);
for i =1:m
    g = g + (x0 - Am(:,i))*(1 - rn(i) / norm(x0 - Am(:,i)));
end

function [H1, tau] = Hx_R_LS(Am, rn,x0)
[n,m] = size(Am);
H1 = zeros(n,n);
tau = 0;
k = 0;
for i = 1:m
    k = x0 - Am(:,i);
    H1 = H1 + (rn(i) / norm(k)^3) * k * k';
    tau = tau + rn(i)/norm(k);
end
tau = m - tau;
 \end{lstlisting}

\section{Iterative Re-Weighting LS  for Source Localization Using Range-Difference Measurements}
\subsection{IRWSRD-LS Algorithm}

The followng m-files implement the iterative re-weighting range-difference based least squares (IRWSRD-LS) algorithm developed in Sec. 2.2 of the thesis. 

\phantom{m}

\noindent
\textbf{Input parameters}

\noindent
$\mbox{\textbf{Am}} = [\Ba_1 \ \Ba_2 \ldots \Ba_m]$, matrix of sensor locations, where $\Ba_i$ is a column vector that stores the coordinates of the $i$th sensor.

\noindent
$\Bd_n = [d_1 \ d_2 \ldots d_m]^T$, noisy range-difference measurements, where $d_i$ is a measurement obtained from the pair of $i$th and the refference sensors.

\noindent
\textbf{Tunable parameters}

\noindent
$Kmax$, number of iterations for re-weighting procedure.

\noindent
$epsi1$, convergence tolerance for iteratively-reweighted solution.

\noindent
$epsi$, convergence tolerance for the bisection routine.

\noindent
\textbf{Otput parameters}

\noindent
$\Bx$, estimated location of the source.

\phantom{m}

\begin{lstlisting}
function x = findroots_w_m(Am,dn,Kmax,epsi1,epsi)
xk = findroots(Am,dn,epsi);
C = [eye(2) zeros(2,1); zeros(1,2) -1];
m = size(Am,2);
err = 10;
ep = 1e-6;
k = 1;
while err >= epsi1 && k <= Kmax,
  nxk = norm(xk);
  for i = 1:m,
      ni = abs(norm(xk-Am(:,i))+nxk+dn(i));
      if ni >= ep,
         wk(i) = 1/ni;
      else
         wk(i) = 1/ep;
      end
  end
  wk = sqrt(m)*wk/norm(wk);
  Wk = diag(wk);
  B = Wk*[-2*Am' -2*dn];
  g = zeros(m,1);
  for i = 1:m,
      g(i) = wk(i)*(dn(i)^2 - norm(Am(:,i))^2);
  end
  btg = B'*g;
  B1 = B'*B;
  P1 = sqrtm(inv(B1));
  W = P1*C*P1;
  [V,D] = eig(W);
  L = diag(D);
  alp = sort(-1./L,'descend');
  alp0 = alp(1);
  alp1 = alp(2);
  alp2 = alp(3);
  aU = alp0;
  aL = alp1;
  dt = aU - aL;
  while dt > epsi,
        lam = 0.5*(aL + aU);
        yt = inv(B1+lam*C)*btg;
        phit = yt'*C*yt;
        if phit > 0,
         aL = lam;
        else
           aU = lam;
        end
        dt = aU - aL;
  end
  lam = 0.5*(aL + aU);
  yt = inv(B1+lam*C)*btg;
  yn1 = yt(3);
  if yn1 >= 0,
     xk_new = yt(1:2);
  else
     P = P1*V;
     f = P'*btg;
     f1 = f(1); f2 = f(2); f3 = f(3);
     f1s = f1^2; f2s = f2^2; f3s = f3^2;
     d1 = D(1,1); d1s = d1^2;
     d2 = D(2,2); d2s = d2^2;
     d3 = D(3,3); d3s = d3^2;
     a0 = f1s*d1 + f2s*d2 + f3s*d3;
     a1 = 2*f1s*d1*(d2+d3) + 2*f2s*d2*(d1+d3) + 2*f3s*d3*(d1+d2);
     a2 = f1s*d1*(d2s+d3s+4*d2*d3) + f2s*d2*(d1s+d3s+4*d1*d3) ...
     + f3s*d3*(d1s+d2s+4*d1*d2);
     a3 = 2*f1s*d1*(d2s*d3+d3s*d2) + 2*f2s*d2*(d1s*d3+d3s*d1) ...
      + 2*f3s*d3*(d1s*d2+d2s*d1);
     a4 = f1s*d1*d2s*d3s + f2s*d2*d1s*d3s + f3s*d3*d1s*d2s;
     b_lu = [a4 a3 a2 a1 a0];
     rts = roots(b_lu);
     lamq = [];
     for i = 1:4,
         if imag(rts(i)) == 0,
            lamq = [lamq real(rts(i))];
         end
     end
     I02 = [];
     I1 = [];
     Lq = length(lamq);
     for i = 1:Lq,
         ti = lamq(i);
         if ti > alp0,
            I02 = [I02 ti];
         elseif ti < alp0 && ti > alp1,
            I1 = [I1 0];
         elseif ti < alp1 && ti > alp2,
            I02 = [I02 ti];
         end
     end
     L02 = length(I02);
     Yt2 = zeros(3,1);
     for i = 1:L02,
         yi = inv(B1+I02(i)*C)*btg;
         if yi(3) >= 0,
            Yt2 = [Yt2 yi];
         end
     end
     L2s = size(Yt2,2);
     obj2 = zeros(L2s,1);
     for i = 1:L2s,
         obj2(i) = norm(B*Yt2(:,i) - g);
     end
     [ymin2,ind2] = min(obj2);
     yt = Yt2(:,ind2);
     xk_new = yt(1:2);
  end
  err = norm(xk_new - xk);
  xk = xk_new;
  k = k + 1;
end
x = xk;
\end{lstlisting}

\subsection{Hybrid IRWSRD-LS}

The followng m-files implement the \textit{hybrid} iterative re-weighting range-difference based least squares (IRWSRD-LS) algorithm developed in Sec. 2.2 of the thesis. 

\phantom{m}

\noindent
\textbf{Input parameters}

\noindent
$fname$, name of the function to calculate the value of the LS objective at a point.

\noindent
$gname$, name of the function to calculate the gradient of the LS objective at a point.

\noindent
$hname$, name of the function to calculate the Hessian of the LS objective at a point.


\noindent
$\Bx_0$, solution of the $\mbox{findroots\_w\_m.m}$ as an initial point for the hybrid IRWSRD-LS algorithm.


\noindent
$\mbox{p} = [m \ d_1 \ d_2 \ldots d_m \ \Ba_1^T \ \Ba_2^T \ldots \Ba_m^T]^T$, a column vector of sensor and measurement data, where $m$ is the number of sensors, $d_i$ is a noisy range-difference measurement obtained from the pair of $i$th and the refference sensors, and $\Ba_i$ is a column vector that stores the coordinates of the $i$th sensor.


\noindent
\textbf{Tunable parameters}

\noindent
$opt$, a parameter allowing the user to choose the implementation of the inexact line search, $opt = 1$ toggles the algorithm to use inexact line search by backtracking, $opt = 2$ toggles the algorithm to use Fletcher's inexact line search.

\noindent
$epsi$, convergence tolerance.

\noindent
\textbf{Otput parameters}

\noindent
$\Bx_s$, estimated location of the source.

\noindent
$f_s$, value of the LS objective at the estimated soluion point.

\noindent
$k$, number of iterations it took the algorithm to  converge to the solution point with the convergence tolerance $epsi$.

\noindent
\textbf{Calling sequence}

\noindent
Function $\mbox{newton\_RD.m}$ calls $\mbox{inex\_lsearch.m}$ or $\mbox{bt\_lsearch.m}$ to perform the line search, $\mbox{Hx\_RD\_LS.m}$ to evaluate the Hessian of the LS objective at a point, $\mbox{fx\_RD\_LS.m}$ and $\mbox{gx\_RD\_LS.m}$ to evaluate the objective and the gradient of the LS objective at a point respectively.  

\noindent
Inexact line search by backtracking $\mbox{bt\_lsearch.m}$ and Fletcher's inexact line search $\mbox{inex\_lsearch.m}$ are not included in listing, as the implementation is not original. 


\phantom{m}

\begin{lstlisting}
function [xs,fs,k] = newton_RD(fname,gname,hname,x0,p,epsi,opt)
k = 1;
xk = x0;
gk = feval(gname,xk,p);
Hk = feval(hname,xk,p);
dk = -inv(Hk)*gk;
if opt == 1
    ak = bt_lsearch(xk,dk,fname,gname,p);
elseif opt == 2
    ak = inex_lsearch(xk,dk,fname,gname,p);
end
adk = ak*dk;
gn = norm(gk);
while gn > epsi,
  xk = xk + adk;
  gk = feval(gname,xk,p);
  Hk = feval(hname,xk,p);
  dk = -inv(Hk)*gk;
    if opt == 1
        ak = bt_lsearch(xk,dk,fname,gname,p);
    elseif opt == 2
        ak = inex_lsearch(xk,dk,fname,gname,p);
    end
  adk = ak*dk;
  gn = norm(gk);
  k = k + 1;
end
xs = xk + adk;
fs = feval(fname,xs,p);
\end{lstlisting} 

\begin{lstlisting}
function f = fx_RD_LS(x,p)
p = p(:);
m = p(1);
dn = p(2:(m+1));
am = p(m+2:end);
Am = vec2mat(am,2)';
f = 0;
for i = 1:m,
    si = Am(:,i);
    fi = norm(x-si)- norm(x) - dn(i);
    f = f + fi^2;
end
f = 0.5*f;
\end{lstlisting} 

\begin{lstlisting}
function g = gx_RD_LS(x,p)
p = p(:);
m = p(1);
dn = p(2:(m+1));
am = p(m+2:end);
Am = vec2mat(am,2)';
nx = norm(x);
g = [0 0]';
for i = 1:m,
    si = Am(:,i);
    ri = dn(i);
    ni = norm(x-si);
    if nx > 0,
       xt = x/nx;
       if ni > 0,
          qi = (x - si)/ni;
          ci = ni - nx - ri;
          gi = qi - xt;
       elseif ni == 0,
          ci = -nx - ri;
          gi = -xt;
       end
    elseif nx == 0,
       if ni > 0,
          qi = (x - si)/ni;
          ci = ni - ri;
          gi = qi;
       elseif ni == 0,
          ci = -ri;
          gi = zeros(2,1);
       end
    end
    g = g + ci*gi;
end
\end{lstlisting} 

\begin{lstlisting}
function H = Hx_RD_LS(x,p)
p = p(:);
m = p(1);
dn = p(2:(m+1));
am = p(m+2:end);
Am = vec2mat(am,2)';
Hw = zeros(2,2);
nx = norm(x);
I = eye(2,2);
for i = 1:m,
    si = Am(:,i);
    ri = dn(i);
    ni = norm(x-si);
    ci = ni - nx - ri;
    if nx > 0,
       xt = x/nx;
       H2 = (I - xt*xt')/nx;
       if ni > 0,
          qi = (x - si)/ni;
          hi = (qi - xt);
          H3i = (I - qi*qi')/ni;
          Hi = hi*hi' + ci*(H3i - H2);
       elseif ni == 0,
          Hi = xt*xt' - ci*H2 ;
       end
    elseif nx == 0,
       if ni > 0,
          qi = (x - si)/ni;
          H3i = (I - qi*qi')/ni;
          Hi = qi*qi' + ci*H3i;
       elseif ni == 0,
          Hi = zeros(2,2);
       end
    end
    Hw = Hw + Hi;
end
[V,D] = eig(Hw);
d = diag(D);
dm = max(d,5e-2);
H = V*diag(dm)*V';
\end{lstlisting} 


\section{PCCP-Based LS Algorithm}

The followng m-files implement the constrained penalty convex-concave procedure for solving the localization problem based on range measurements. The implementation follow the alrorithm developed in Chapter 3.

\phantom{m}

\noindent
\textbf{Input parameters}

\noindent
$\mbox{\textbf{Am}} = [\Ba_1 \ \Ba_2 \ldots \Ba_m]$, matrix of sensor locations, where $\Ba_i$ is a column vector that stores the coordinates of the $i$th sensor.

\noindent
$\Br_n = [r_1 \ r_2 \ldots r_m]^T$, noisy range measurements, where $r_i$ is a measurement obtained from the $i$th sensor.

\noindent
\textbf{Tunable parameters}

\noindent
$gam$ a parameter that controls the bounds in the constraints.

\noindent
$\Bx_0$, initial point for the source location.

\noindent
$K$, number of iterations for re-weighting procedure.

\noindent
\textbf{Otput parameters}

\noindent
$\Bx_w$, estimated location of the source.

\phantom{m}

\begin{lstlisting}
function [x_w] = pccp(Am,rn,gam,x0,K)
[n,m] = size(Am);
k = 0;
xk = x0;
rnb_p = rn + gam*sig;
rnb_n = rn - gam*sig;
ab = (mean(Am'))';
tau = 1;
tau_max = 100000;
mi = 1/m;
m2 = 2*m;
nn = 1000;
while (k < K) 
    cvx_begin quiet
       variable x(n)
       variable s(m2);
       v = ab;
       for i = 1:m,
           xai = xk - Am(:,i);
           v = v + (mi*rn(i)/norm(xai))*xai;
       end
       minimize(x'*x - 2*x'*v + tau*sum(s));
       subject to
       for i = 1:m,
           norm(x-Am(:,i)) <= rnb_p(i) + s(i);
           xai = xk - Am(:,i);
           ni = norm(xai);
           xain = xai/ni;
           -xain'*(x-xk) - ni + rnb_n(i) <= s(m+i);
       end
       s >= 0;
    cvx_end
    nn = norm(xk-x);
    xk = x;
    tau = min(1.5*tau,tau_max);
    k = k + 1;
end
xk_w = xk;
\end{lstlisting}

\newpage

\section{SCR-RDLS Algorithm}

The followng m-files implement the sequential convex relaxation for solving the localization problem based on range-difference measurements. The implementation follow the alrorithm developed in Sec.4.1.

\phantom{m}

\noindent
\textbf{Input parameters}

\noindent
$\mbox{\textbf{Am}} = [\Ba_1 \ \Ba_2 \ldots \Ba_m]$, matrix of sensor locations, where $\Ba_i$ is a column vector that stores the coordinates of the $i$th sensor.

\noindent
$\Bd_n = [d_1 \ d_2 \ldots d_m]^T$, noisy range-difference measurements, where $d_i$ is a measurement obtained from the pair of $i$th and the refference sensors.


\noindent
\textbf{Tunable parameters}

\noindent
$bt$, bound for increments of design variables.

\noindent
$epsi$, convergence tolerance.


\noindent
\textbf{Otput parameters}

\noindent
$\Bx_w$, estimated location of the source.

\phantom{m}


\begin{lstlisting}
function x_w = scr_rdls(Am,dn,bt,epsi)
m = length(dn);
em = ones(m,1);
zm = zeros(m,1);
Im = eye(m);
cvx_begin quiet
   variable x0(2);
   variable y(1);
   variable z(m);
   minimize((z-y*em-dn)'*(z-y*em-dn));
   subject to
   norm(x0) <= y;
   for i = 1:m,
       norm(x0-Am(:,i)) <= z(i);
   end
cvx_end
xk = x0;
yk = norm(xk);
zk = zeros(m,1);
for i = 1:m,
    zk(i) = norm(xk-Am(:,i));
end
mu = 20;
P = zeros(m+1,m+3);
q = zeros(m+1,1);
k = 0;
err = 1;
Kmax = 100;
while (err >= epsi && k <= Kmax),
    for i = 1:m,
        vi = xk - Am(:,i);
        nvi = norm(vi);
        q(i) = zk(i) - nvi;
        if nvi < 1e-6,
            gvi = zeros(2,1);
        else
            gvi = vi/nvi;
        end
        P(i,:) = [gvi', 0, -Im(i,:)];
    end
    nx = norm(xk);
    if nx < 1e-6,
       P(m+1,:) = [0 0 -1 zm'];
    else
       P(m+1,:) = [xk'/nx, -1, zm'];
    end
    q(m+1) = yk - nx;
    cvx_begin quiet
       variable dx(2,1)
       variable dy(1,1)
       variable dz(m,1)
       variable u(m+1,1)
       variable v(m+1,1)
       variable w(1,1)
       dtt = [dx; dy; dz];
       pk = dn - zk + yk*em;
       minimize(norm(dz - dy*em - pk) + mu*(sum(u+v)+sum(w)));
       subject to
       P*dtt - q == u - v;
       abs(dtt) <= bt + w;
       u >= 0;
       v >= 0;
       w >= 0;
       mu = min(1.5*mu,1e4);
     cvx_end
     err = norm(dtt);
     xk = xk + dx;
     yk = yk + dy;
     zk = zk + dz;
     k = k + 1;
     % current_status = [k err norm(u) norm(v) norm(w)]
end
x_w = xk;
\end{lstlisting}

\section{SCR-RLS Algorithm}

The followng m-files implement the sequential convex relaxation for solving the localization problem based on range measurements. The implementation follow the alrorithm developed in Sec. 4.2.

\phantom{m}

\noindent
\textbf{Input parameters}

\noindent
$\mbox{\textbf{Am}} = [\Ba_1 \ \Ba_2 \ldots \Ba_m]$, matrix of sensor locations, where $\Ba_i$ is a column vector that stores the coordinates of the $i$th sensor.

\noindent
$\Br_n = [r_1 \ r_2 \ldots r_m]^T$, noisy range measurements, where $r_i$ is a measurement obtained from the $i$th sensor.

\noindent
\textbf{Tunable parameters}

\noindent
$gamma$ a parameter that controls the size of the convex hull that defines a feasibility region of the proble.

\noindent
$\Bx_0$, initial point for the source location.

\noindent
$K$, number of iterations for re-weighting procedure.

\noindent
\textbf{Otput parameters}

\noindent
$\Bx_w$, estimated location of the source.

\phantom{m}

\begin{lstlisting}
function x_w = scr_rls(Am,rn,x0,gamma,K)
m = length(rn);
xk = x0(:);
rn = rn(:);
k = 0;
g0 = 1-gamma;
% quadratically updates the value of \gamma
a = (1-g0)/(K-1)^2;        
% epsi = 1e-8;
while k < K,
    gk = 1 - a*(k-K+1)^2;   
    cvx_begin quiet
      variable x(2,1)
      variable z(m,1)
      minimize(norm(z-rn))
      subject to
      for i = 1:m,
          ai = Am(:,i);
% the arc portion of the convex region 
% is updated as iteration proceeds.
          norm(x-ai) <= (2-gk)*z(i);  
          xai = (xk-ai)/norm(xk-ai);
% the line portion of the convex region 
% is pushed towards the arc portion.
          xai'*(x-ai) >= gk*z(i);     
      end
    cvx_end
    xk = x;
    k = k + 1;
end
x_w = xk;
\end{lstlisting}