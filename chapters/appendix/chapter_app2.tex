\startappendix{Code Listings}
\label{chapter:app2}


%\section{Iterative Re-Weighting LS  for Source Localization Using Range Measurements}
\section{IRWSR-LS}
3.21

%\section{Iterative Re-Weighting LS  for Source Localization Using Range-Difference Measurements}
\section{IRWSRD-LS}
3.32

\section{PCCP-Based LS}

\begin{lstlisting}
 Apply a constrained CCP to the localization problem based on range
 measurements. Both upper-bound and lower-bound constraints are imposed.
 Input:
 Am: Am = [a1 a2 ... am] with ai the location of the ith sensor.
 r: noise-free range measurements r = [r1 r2 ... rm] with ri = norm(xs-ai).
 st: initial state for noise generation.
 sig: standard deviation of measurement noise.
 gam: a parameter that controls the bounds in the constraints.
 x0: initial point for the source location.
 K: number of CCG iterations.
 Output:
 xw: estimated location of the source.
 xw1: estimated location of the source using SR-LS.
 Written by W.-S. Lu, University of Victoria.
 Last modified: Feb. 2, 2015.
 Example: load data_ex1
 [xw,xw1] = pccp(Am,r,1e-2,3,zeros(2,1),20);
 
function [F_ccp,F_rls,solution,slack] = pccp(Am,rn,sig,gam,x0,K)
[n,m] = size(Am);
k = 0;
xk = x0;
rnb_p = rn + gam*sig;
rnb_n = rn - gam*sig;
ab = (mean(Am'))';
tau = 1;
tau_max = 100000;
mi = 1/m;
m2 = 2*m;
nn = 1000;
Xk=[];
while (k < K) %&& (nn >= 1e-5)
    cvx_begin quiet
       variable x(n)
       variable s(m2);
       v = ab;
       for i = 1:m,
           xai = xk - Am(:,i);
           v = v + (mi*rn(i)/norm(xai))*xai;
       end
       minimize(x'*x - 2*x'*v + tau*sum(s));
       subject to
       for i = 1:m,
           norm(x-Am(:,i)) <= rnb_p(i) + s(i);
           xai = xk - Am(:,i);
           ni = norm(xai);
           xain = xai/ni;
           -xain'*(x-xk) - ni + rnb_n(i) <= s(m+i);
       end
       s >= 0;
    cvx_end
    x=x(:);
    Xk = [Xk x];
    nn = norm(xk-x);
    xk = x;
    tau = min(1.5*tau,tau_max);
    k = k + 1;
end
% disp(sprintf('Solution point after %d CCP iterations:',k))
solution = xk;
F_rls = zeros(K,1);
F_ccp = zeros(K,1);
for k = 1:K
    %value of the NNLS objective at the solution point
    sol  = Xk(:,k);
    v = zeros(2,1);
    for i = 1:m,
       xai = sol - Am(:,i);
       v = v + (mi*rn(i)/norm(xai))*xai;
       F_rls(k) = F_rls(k) + (rn(i) - norm(sol - Am(:,i)))^2;
    end
    % disp(sprintf('Final value of the objective function'))
    F_ccp(k) = sol'*sol - 2*sol'*v ;%+ tau*sum(s);
end
% disp('Final values of relaxation parameters:')
 slack = s;
\end{lstlisting}