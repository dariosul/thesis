\startfirstchapter{Introduction}
\label{chapter:introduction}

Tips from Uvic template:

%\pagebreak
\section{My Claims}
State what's new here. Make an impact here. How about something like the following box:

I make \textit{four} claims which my dissertation validates:

\framebox{%
\parbox{5in}{
	My new algorithm to solve the problem of doing nothing include these important new features whose practical applicability can be proved both formally and empirically:
	\begin{enumerate}
	\item first feature;
	\item second feature;
	\item everything is much easier to understand, and therefore, easier to implement correctly.
	\end{enumerate}
}}

\noindent Claim 1 and claim 2 are \textit{quantitative} - they will be proven by experiment.

\noindent Claim 3 is \textit{qualitative} - they will be demonstrated by argument.

\subsection{The Importance of My Claims}

Some very important positive consequences arise from the validation of the above claims.
It is these consequences that comprise a significant positive contribution to research in the field of whatever the field is.

\noindent Claim 1 implies that:
\begin{enumerate}
\item{Something profound which applies to:
	\begin{itemize}
	\item {something excellent;}
	\item {something important.}
	\end{itemize}}
\item{Something else just as profound.}
\end{enumerate}

\noindent Claim 2 implies that:
\begin{itemize}
\item{Repeat as above if necessary and useful.}
\end{itemize}

\noindent The consequence of claim 3 is that:
\begin{itemize}
\item{There must be something good coming out of all this work!}
\end{itemize}

\section{Agenda}

\begin{description}
\item[\textbf{Chapter 1}] contains a statement of
the claims which will be proved by this dissertation followed by an overview of the structure of the document itself.
\item[\textbf{Chapter 2}] describes in details the open problem which is to be tackled together with its context, its impact and the overall motivation for the research overall.
\item[\textbf{Chapter 3}] gives the new research, its methodology, the algorithms involved, the new solution, the new work done. Formal proofs and arguments are made here. This is the first of the two contributions expected in a thesis for a graduate degree.
\item[\textbf{Chapter 4}] is where the experiments and the methodology for them is fully described. The first part includes all details of how the empirical side of the research has been conducted. Note that not every thesis has this empirical portion.
\item[\textbf{Chapter 5}] includes the evaluation of the data presented above and the comparisons with the work of others, to show how much better the new approach is. This is the second of the two contributions expected in a thesis for a graduate degree. Note that this part could be consolidated into the chapter above.
\item[\textbf{Chapter 6}] contains a restatement of the claims and results of the dissertation. It also enumerates avenues of future work for further development of the concept and its applications.
\end{description}
