\startfirstchapter{Introduction}
\label{chapter:introduction}

%Tips from Uvic template:
%
%%\pagebreak

%State what's new here. Make an impact here. How about something like the following box:
%
%I make \textit{four} claims which my dissertation validates:
%
%\framebox{%
%\parbox{5in}{
%	My new algorithm to solve the problem of doing nothing include these important new features whose practical applicability can be proved both formally and empirically:
%	\begin{enumerate}
%	\item first feature;
%	\item second feature;
%	\item everything is much easier to understand, and therefore, easier to implement correctly.
%	\end{enumerate}
%}}
%
%\noindent Claim 1 and claim 2 are \textit{quantitative} - they will be proven by experiment.
%
%\noindent Claim 3 is \textit{qualitative} - they will be demonstrated by argument.
%
%\subsection{The Importance of My Claims}
%
%Some very important positive consequences arise from the validation of the above claims.
%It is these consequences that comprise a significant positive contribution to research in the field of whatever the field is.
%
%\noindent Claim 1 implies that:
%\begin{enumerate}
%\item{Something profound which applies to:
%	\begin{itemize}
%	\item {something excellent;}
%	\item {something important.}
%	\end{itemize}}
%\item{Something else just as profound.}
%\end{enumerate}
%
%\noindent Claim 2 implies that:
%\begin{itemize}
%\item{Repeat as above if necessary and useful.}
%\end{itemize}
%
%\noindent The consequence of claim 3 is that:
%\begin{itemize}
%\item{There must be something good coming out of all this work!}
%\end{itemize}
%
%%\section{Agenda}
%
%\begin{description}
%\item[\textbf{Chapter 1}] contains a statement of
%the claims which will be proved by this dissertation followed by an overview of the structure of the document itself.
%\item[\textbf{Chapter 2}] describes in details the open problem which is to be tackled together with its context, its impact and the overall motivation for the research overall.
%\item[\textbf{Chapter 3}] gives the new research, its methodology, the algorithms involved, the new solution, the new work done. Formal proofs and arguments are made here. This is the first of the two contributions expected in a thesis for a graduate degree.
%\item[\textbf{Chapter 4}] is where the experiments and the methodology for them is fully described. The first part includes all details of how the empirical side of the research has been conducted. Note that not every thesis has this empirical portion.
%\item[\textbf{Chapter 5}] includes the evaluation of the data presented above and the comparisons with the work of others, to show how much better the new approach is. This is the second of the two contributions expected in a thesis for a graduate degree. Note that this part could be consolidated into the chapter above.
%\item[\textbf{Chapter 6}] contains a restatement of the claims and results of the dissertation. It also enumerates avenues of future work for further development of the concept and its applications.
%\end{description}

\section{The Localization Problem}
\label{chapter:problem}


TODO

Review of ranging and localization methods, theory behind it, application, limitations. 

TOA,
 
TDOA,

AOA ?,

non-range-based? 

"Geolocation techniques"

something like:

the material developed here / mathematical tools and methods are suitable for many different scenarios 
OR
can have different world life applications, for example: TOA, TDOA, static positioning using UWB range measurements. 

%UVic thesis template tips:
%
%
%why the problem is important, what its impact is and what its application, if any. Here you are free to elaborate and write as much as you think is necessary to avoid the examination doubt that you have a brilliant new solution to a trivial and unimportant issue.
%
%"The Craft of Research" by Wayne Booth \cite{booth1}, which can be found in the main library at Q180.55 M4B66. From there I have transferred to my writing a fairly simple structure for talking about the topic of the research, with the question to be asked and its motivation and significance. It goes as follows:
%\begin{enumerate}
%\item {\textit{I am trying to learn about (working on, studying...)}}
%\item {\textit{because I want to find out....}}
%\item {\textit{in order to understand...}}
%\end{enumerate}
%
%Another way of looking at this is to ask the
%\textit{what}, \textit{why} and \textit{where}, starting from a \textit{setting} of the problem with a first point A, stating what the \textit{goal} is at point B and having an \textit{action link} between the two which will encompass your new solution. As surprising as this may be to some of you, I found reading a book from Microsoft very useful: "Beyond Bullet Points: Using Microsoft Office PowerPoint 2007 to Create Presentations That Inform" \cite {atkin}. The goal of the book is to improve presentations with Power Point, but there is a lot that can be transferred towards \textit{effective communication} for a thesis.
%
%In summary, my view of the second chapter on
%\textit{"The Problem to be solved"} is as follows:
%\begin{enumerate}
%\item {\textit{Not} all the background and definitions (boring!) - use instead just-in-time explanations as needed in every context as it comes up;}
%\item {Motivation in depth;}
%\item {Tutorial high level explanation, where it is important to choose the right pitch: who is the audience? who are you teaching here?}
%\item {Make it exciting, make it current, make it important - why do I want to keep reading?}
%\item {Should you list here the solutions from other researchers? I think not, list instead the different facets of the problems that other researchers have attacked.}
%\item {A taxonomy can be extremely useful to place your problem and its particular special features within the perfect context of the overall area, as you need to make sure that the reader understands perfectly what you are trying to solve.}
%\end{enumerate}

\section{Contributions and Organization of the Thesis}

\subsection{Contributions  of the Thesis}

\subsection{Organization of the Thesis}