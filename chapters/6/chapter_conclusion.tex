\startchapter{Conclusions and Future Work}

In this thesis we studied the solution methods for the single source localization. In particular, we focused on the ranging techniches, namely range and range-difference based methods. For both types of measurements, we used the nonlinear least squares formulations to estimate the source location. 

In Chapter 2, we turn our attention to the global methods for solving the approximation of the nonlinear least squares problem. We begin by
transforming the SR-LS and SRD-LS algorithms \cite{BeckStLi} into an
iterative procedure. At each iteration the algorithms find a global solution to an  adaptively weighted SR-LS (SRD-LS) subproblem that, as iterations proceed, approximates closely the LS solution.Numerical results are presented to illustrate the proposed algorithms in comparison with some state-of-the-art methods \cite{IRWSg}.


In Chapter 3,  a new iterative method for locating a radiating source based on noisy range measurements have been proposed. The method is developed by transforming the original least-squares problem to a difference-of-convex-functions programming problem which is in turn relaxed to a sequential convex minimization based on PCCP that can be efficiently solved with an infeasible initial point. By imposing additional constraints, we  enforce the iteration path towards the LS solution. Several strategies to secure a good initial point are also provided.  Along the way, we see that CCP allows a natural embedding of the LS formulation for localization into a sequential convex formulation in a sense that few additional terms and functions are introduced into the procedure \cite{PCCP}. Simulation results demonstrate promising localization performance when compared with some best known results from the literature.

In Chapter 4, we revisit the localization problem based on range-difference measurement. We first convert the unconstrained LS problem to constrained LS and develop an iterative method for solving the constrained problem.  The central part of the procedure is a convex quadratic programming (QP) problem that needs to be solved in each iteration to provide an increment vector that updates the present iterate to next towards the solution of the localization problem at hand. Later in the chapter, we apply a similar technique to the range-based localization problem, where at each iteration we solve a relaxed SOCP problem.


This thesis is focused on the single source localization problem. 

An interesting work has been presented on multiple source localization in wireless sensor networks based on compressive sensing cite(LinCai, LiuCS)  

However the  
Multi-source localization, when the range measurements were obtained from the energy-based readings. 

Another area of future interest is study and mitigation of the influense of sensor geometry on the accuracy of the developed methods. In any ranging localization system, the geometry of the sensor network  affects position precision.  When evaluating the performance of the new positioning algorithms One such metric, called the Geometric Dilusion of Precision, has attracted a lot of research interest cite(9-12), specifically for positioning in indoor  environments. DGOP 

methods. For example, one of the highest impacts on accuracy degradation is Geometric Dilusion of Presicion. Optimum sensor arrangement is a separate research area. 

Geometric dilution of precision (GDOP) can be used to measure the positioning precision of the localization system. 
Sensor node's contribution to the overall GDOP value is adopted as the evaluation criteria. The nodes whose contribution value is greater than the threshold will be selected. 
GDOP is defined as the ratio between the error of ranging measurement and that of localization. Since it can separate the geometry factor out of the localization error, GDOP shows superior property for WSN application. 



    Levanon N.
Lowest GDOP in 2-D scenariosIEE Proceedings: Radar, Sonar and Navigation200014731491552-s2.0-0033704277doi:10.1049/ip-rsn:20000322
CrossRefGoogle Scholar


    Sharp I., Yu K., Guo Y. J.
GDOP analysis for positioning system designIEEE Transactions on Vehicular Technology2009587337133822-s2.0-69549105982doi:10.1109/TVT.2009.2017270
CrossRefGoogle Scholar


    Sharp I., Yu K., Hedley M.
On the GDOP and accuracy for indoor positioningIEEE Transactions on Aerospace and Electronic Systems2012483203220512-s2.0-84863926833doi:10.1109/TAES.2012.6237577
CrossRefGoogle Scholar


    Chu L.-C., Tseng P.-H., Feng K.-T.
GDOP-assisted location estimation algorithms inwireless location systemsProceedings of the IEEE Global Telecommunications Conference (GLOBECOM ′08)December 2008New Orleans, La, USA540454082-s2.0-67249160795doi:10.1109/GLOCOM.2008.ECP.1032
Google Scholar
