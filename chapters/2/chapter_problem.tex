\startchapter{The Problem to be Solved}
\label{chapter:problem}


TODO

Review of ranging and localization methods, theory behind it, application, limitations: 

TOA,
 
TDOA,

AOA ?,

non-range-based? 
%
%"Geolocation techniques"
%
%UVic thesis template tips:
%
%
%why the problem is important, what its impact is and what its application, if any. Here you are free to elaborate and write as much as you think is necessary to avoid the examination doubt that you have a brilliant new solution to a trivial and unimportant issue.
%
%"The Craft of Research" by Wayne Booth \cite{booth1}, which can be found in the main library at Q180.55 M4B66. From there I have transferred to my writing a fairly simple structure for talking about the topic of the research, with the question to be asked and its motivation and significance. It goes as follows:
%\begin{enumerate}
%\item {\textit{I am trying to learn about (working on, studying...)}}
%\item {\textit{because I want to find out....}}
%\item {\textit{in order to understand...}}
%\end{enumerate}
%
%Another way of looking at this is to ask the
%\textit{what}, \textit{why} and \textit{where}, starting from a \textit{setting} of the problem with a first point A, stating what the \textit{goal} is at point B and having an \textit{action link} between the two which will encompass your new solution. As surprising as this may be to some of you, I found reading a book from Microsoft very useful: "Beyond Bullet Points: Using Microsoft Office PowerPoint 2007 to Create Presentations That Inform" \cite {atkin}. The goal of the book is to improve presentations with Power Point, but there is a lot that can be transferred towards \textit{effective communication} for a thesis.
%
%In summary, my view of the second chapter on
%\textit{"The Problem to be solved"} is as follows:
%\begin{enumerate}
%\item {\textit{Not} all the background and definitions (boring!) - use instead just-in-time explanations as needed in every context as it comes up;}
%\item {Motivation in depth;}
%\item {Tutorial high level explanation, where it is important to choose the right pitch: who is the audience? who are you teaching here?}
%\item {Make it exciting, make it current, make it important - why do I want to keep reading?}
%\item {Should you list here the solutions from other researchers? I think not, list instead the different facets of the problems that other researchers have attacked.}
%\item {A taxonomy can be extremely useful to place your problem and its particular special features within the perfect context of the overall area, as you need to make sure that the reader understands perfectly what you are trying to solve.}
%\end{enumerate}

