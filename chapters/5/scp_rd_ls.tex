\section{LQP}

We begin by re-writing the unconstrained problem in (\ref{eq:6.2}) as a constrained problem with second-order cone constraints%can be equivalently written as
\setcounter{abc}{0}
\begin{eqnarray} \label{eq:6.3} %1.7
\stepcounter{abc}
\setcounter{equation}{7}
\Min_{\Bx, y, \Bz} & & \sum^m_{i=1}\left( z_i - y - d_i\right)^2\\
\stepcounter{abc}
\setcounter{equation}{7}
\mbox{subject to:}& &\|\Bx - \Ba_i\| = z_i, \quad  i = 1, 2, \ldots m \\
\stepcounter{abc}
\setcounter{equation}{7}
& &\|\Bx\|  = y
\end{eqnarray}
Let $\tilde{\Bx} = [\Bx^T \ y \ z_1 \ldots z_m]^T$ with $\Bx \in R^n, y \in R, \Bz \in R^m$ be a known strictly feasible point of the problem in (\ref{eq:6.3}). 
%\setcounter{abc}{0}
%\begin{eqnarray} 
%\nonumber
%\Min_{\Bx, y, \Bz} & & \sum^m_{i=1}\left( z_i - y - d_i\right)^2 \\
%\nonumber
%\mbox{subject to:}& &\|\Bx - \Ba_i\| \leq z_i \\
%\nonumber
%& &\|\Bx\|  \leq y, \quad  i = 1, 2, \ldots m
%\end{eqnarray}
Let $\tilde{\Bdelta} = [\Bdelta_x^T \  \delta_y \ \delta_{z_1} \ \ldots \  \delta_{z_m}]^T$, $\Bdelta_x \in R^n, \Bdelta_y \in R, \Bdelta_z \in R^m$ be a set of small perturbations to the above variables such that $|\tilde{\Bdelta}| \leq \beta\symb{1_{m+3}}$ where $\beta > 0$ is a small positive constant. We seek to find an increment vector $\tilde{\Bdelta} = [\Bdelta_x^T \  \delta_y \ \Bdelta_{z}^T]^T$ such that the next iterate
\setcounter{abc}{0}
\begin{eqnarray} \label{eq:6.4} %1.8
\stepcounter{abc}
\Bx^{k+1} = \Bx^k + \Bdelta_x \\
\stepcounter{abc}
\setcounter{equation}{8}
y^{k+1} = y^k + \delta_y \\
\stepcounter{abc}
\setcounter{equation}{8}
\Bz^{k+1} = \Bz^k + \Bdelta_z
\end{eqnarray}
reduces the objective function in (\ref{eq:6.4}a) and, at the same time, better satisfies the constraints in (\ref{eq:6.4}b) and (\ref{eq:6.4}c). At the $(k+1)$th iteration with $\tilde{\Bx}^k$ \textit{known}, %update $(\Bx^k, y^k, \Bz^k)$ to 
%Substituting \ref{eq:6.4} in \ref{eq:6.3}
 the objective in \ref{eq:6.3} can be written as
 \setcounter{abc}{0}
\begin{eqnarray} \label{eq:6.5} %1.9
F(\hat{\Bx}) & =  & \sum^m_{i = 1} \left(z^k_i + \delta_{z_i} - (y^k + \delta_y) - d_i \right)^2 \\
\nonumber
& = & \sum^m_{i = 1} \left(- \delta_y + \delta_{z_i}  - \tilde{d_i^k} \right)^2
\end{eqnarray}
where 
\begin{equation}
\nonumber
\tilde{d}^k_i =  d_i - y^k - z_i^k
\end{equation}
 are grouped known constant terms. By substituting (\ref{eq:6.4}b) into (\ref{eq:6.3}b), the constraints in (\ref{eq:6.3}b) become
\begin{equation}
\nonumber
\|\Bx^k + \Bdelta^k_x - \Ba_i\| = z^k_i +\delta_{z_i}, \quad i = 1, 2, \ldots, m
\end{equation}
These constraints can be convexified by squaring both sides of the equality and then re-grouping the terms on the left-hand side as follows
\begin{eqnarray}
\nonumber
\|(\Bx^k  - \Ba_i) + \Bdelta^k_x\|^2 & = & \left(z^k_i +\delta_{z_i}\right)^2 \\
\nonumber
\Leftrightarrow 
\|\Bx^k  - \Ba_i\|^2 + 2\left( \Bx^k  - \Ba_i \right)^T\Bdelta_x + \cancelto{\mbox{\scriptsize{0}}}{\|\Bdelta_x\|^2}  & = & \left(z_i^k\right)^2 + 2z_i^k\delta_{z_i} + \cancelto{\mbox{\scriptsize{0}}}{\Bdelta^2_{z_i}} \\
\nonumber
\Leftrightarrow \|\Bx^k  - \Ba_i\|^2 + 2\left( \Bx^k  - \Ba_i \right)^T\Bdelta_x   & \approx & \left(z_i^k\right)^2 + 2z_i^k\delta_{z_i} \\
\nonumber
- 2\left( \Bx^k  - \Ba_i \right)^T\Bdelta_x + 2z_i^k\delta_{z_i}  & \approx & \|\Bx^k  - \Ba_i\|^2 - \left(z_i^k\right)^2
\end{eqnarray}
The constraint in (\ref{eq:6.3}c) can be treated in a similar way as
\begin{eqnarray}
\nonumber
\|\Bx^k + \Bdelta_x\| & = & y^k + \delta_y \\
\nonumber
\|\Bx^k + \Bdelta_x\|^2 & = & \left(y^k + \delta_y \right)^2 \\
\nonumber
\Leftrightarrow \|\Bx^k\|^2 + 2\Bdelta_x^T\Bx^k + \cancelto{\mbox{\scriptsize{0}}}{\|\Bdelta_x\|^2} & = & \left(y^k\right)^2 + 2y^k\delta_y +  \cancelto{\mbox{\scriptsize{0}}}{\delta_y^2} \\
\nonumber
\Leftrightarrow -2\left(\Bx^k\right)^T\delta_x + 2y^k\delta & \approx & \|\Bx\|^2 - \left(y^k\right)^2
\end{eqnarray}
The problem in \ref{eq:6.3} can now be written in terms of the \textit{known} feasible  iterate $\tilde{\Bx}^k$ and its \textit{unknown} increment  $\tilde{\Bdelta} = [\Bdelta_x^T \  \delta_y \ \Bdelta_{z}^T]^T$ as
\setcounter{abc}{0}
\begin{eqnarray} \label{eq:6.6} %1.10
\stepcounter{abc}
\Min_{\Bdelta_x, \delta_y, \Bdelta_z}& &\sum^m_{i=1}\left( -\delta_y + \delta_{z_i} -\tilde{d_i}^k\right)^2\\
\setcounter{equation}{10}
\stepcounter{abc}
\mbox{subject to:}& &-2\left(\Bx^k\right)^T\delta_x + 2y^k\delta = \|\Bx^k\|^2 - \left(y^k\right)^2 \\
\setcounter{equation}{10}
\stepcounter{abc}
& &- 2\left( \Bx^k  - \Ba_i \right)^T\Bdelta_x + 2z_i^k\delta_{z_i}  = \|\Bx^k  - \Ba_i\|^2 - \left(z_i^k\right)^2, \\
\nonumber
& &\qquad \qquad \qquad \qquad \qquad \qquad \qquad  \qquad i = 1, 2, \ldots m \\
\setcounter{equation}{10}
\stepcounter{abc}
& &|{\tilde{\Bdelta}}|  \leq \beta\symb{1_{m+3}}
\end{eqnarray}
Clearly, (\ref{eq:6.6}) is a convex QP problem which can be written more
compactly as follows:
\setcounter{abc}{0}
\begin{eqnarray} \label{eq:6.7} %1.11
\stepcounter{abc}
\Min_{{\Bdelta_x, \delta_y, \Bdelta_z}}& &\| -\delta_y\symb{1_m} + \Bdelta_z - \tilde{\Bd_k} \|_2 
\\ 
\setcounter{equation}{11}
\stepcounter{abc}
\mbox{subject to:}& &\BC_k\tilde{\Bdelta}  = \Bp_k \\
\setcounter{equation}{11}
\stepcounter{abc}
 & &|\tilde{\Bdelta}|  \leq \beta \symb{1_{m+3}}
\end{eqnarray}
where
\setcounter{abc}{0}
\begin{equation} \label{eq:6.8} %12
\tilde{\Bd_k} = 
\begin{bmatrix}
d_1 + y^k - z_1^k \\
d_2 + y^k - z_2^k \\
\vdots \\
d_m + y^k - z_m^k \\
\end{bmatrix}, 
\quad \Bp_k = \begin{bmatrix}
\|\Bx^k\|^2 -\left(y^k\right)^2  \\
\|\Bx^k - \Ba_1\|^2 -\left(z_1^k\right)^2 \\
\vdots \\
\|\Bx^k - \Ba_m\|^2 -\left(z_m^k\right)^2 \\
\end{bmatrix}
\end{equation}
\begin{equation}
\nonumber
\BC_k = \begin{bmatrix}
-2\left(\Bx^k\right)^T & 2y^k & 0 & \hdots & 0 \\
-2\left(\Bx^k - \Ba_1\right)^T & 0 & 2z^k_1 & \hdots & 0 \\
\vdots & \vdots & \vdots & \ddots & \vdots \\
-2\left(\Bx^k - \Ba_m\right)^T & 0 & 0 & \hdots & 2z_m^k
\end{bmatrix},
\quad \tilde{\Bdelta} = \begin{bmatrix}
\Bdelta_x \\
\delta_y \\
\Bdelta_z
\end{bmatrix}
\end{equation}
\textbf{starting form this point: "Page 57
• Darya, the text (before Section 4.2.3) right after the first equation at the top needs to be revised. You can do so using some part of my write-up with necessary modification for notation consistency."}
A technical problem making the formulation in (\ref{eq:6.7}) difficult to implement is that it requires a feasible initial point $\Bx^k$. The iterate $\tilde{\Bx}^k$ is known and  strictly feasible  but it is not guaranteed that the $\tilde{\Bx}^{k+1}$ will also be feasible. The problem can be overcome by introducing nonnegative slack variable $s$ into the constraints in \ref{eq:6.7}c to replace their right-hand sides by relaxed upper bounds (as these new bounds themselves are nonnegative variables). To allow non-feasible increments $\tilde{\Bdelta}$, the problem can be overcome by introducing a
nonnegative slack variable $s \geq 0$ \ref{eq:6.8}c to replace their right-hand sides  by a  relaxed upper bound (as this new bound itself is a nonnegative variable). We allow the constraint in $|\tilde{\Bdelta}|  \leq \beta \symb{1_{m+3}}$ be violated so that the  region with lower value of the objective function can be found. 
This leads to a following sequential relaxation step for the problem in \ref{eq:6.3}
\setcounter{abc}{0}
\begin{eqnarray} \label{eq:6.9} %13
\stepcounter{abc}
\Min_{\Bdelta_x, \delta_y, \Bdelta_z, s}& &\| -\delta_y\symb{1_m} + \Bdelta_z - \tilde{\Bd_k} \|_2 + \mu_ks
\\ 
\stepcounter{abc}
\setcounter{equation}{13}
\mbox{subject to:}& &\BC_k\tilde{\Bdelta}  = \Bp_k 
\\
\stepcounter{abc}
\setcounter{equation}{13}
 & &|\tilde{\Bdelta}|  \leq \left(\beta + s\right)\symb{1_{m+3}} 
 \\
\stepcounter{abc}
\setcounter{equation}{13}
& & s \geq 0
\end{eqnarray}
where the weight $\mu_k \geq 0$ increases as iterations proceed until it reaches an upper limit $\mu_{max}$. By using a monotonically increasing $\mu_k$ for the penalty term in \ref{eq:6.9}a, the algorithm reduces the slack variable $s$  very quickly. As a result, new iterates
quickly become feasible as $s$  vanishes. The upper limit $\mu_{max}$ is imposed to avoid numerical difficulties that may occur if $\mu_{k}$ becomes too large and to ensure convergence if a feasible region is not found.


\subsection{The Algorithm}

Based on the analysis above, the localization algorithm for range-difference measurements can be outlined as follows.

%The constraint $\beta$ was imposed on each element of the vector $\tilde{\Bdelta}$ to guarantee that at each iteration is sufficiently small.

%Dropping the constraints in \ref{eq:6.8}f,g allows more variety in choosing the search direction, which increases the likelihood of the algorithm not to get trapped in the local minimimum.

%\phantom{m}
%\framebox{%
%\parbox{5.4in}{
\label{alg:socp_rd}
%\phantom{m}

\noindent \textbf{Algorithm 4. }

%\phantom{m}

1) Input data: Sensor locations $\{\Ba_i, i=1,\ldots,m\}$, range-difference measurements $\{d_i, i=1,\ldots,m\}$, initial point $\Bx_0$, maximum number of iterations $K_{max}$, initial weight $\mu_0$ and upper limit of weight $\mu_{max}$, increment bound $\beta$. %\gamma, \sigma$, 
Set iteration count to $k = 0$. Form

\begin{equation} 
\setcounter{abc}{0}
\nonumber
y_0 = \|\Bx_0\| ,\quad
\Bz_0 = \begin{bmatrix}
\|\Bx_0 - \Ba_1\|^2  \\
\|\Bx_0 - \Ba_2\|^2  \\
\vdots \\
\|\Bx_0 - \Ba_m\|^2  \\
\end{bmatrix}
\end{equation}

%\phantom{m}

2) Form $\tilde{\Bd_k}, \Bp_k$ and $\BC_k $ as 
\begin{equation} 
\setcounter{abc}{0}
\nonumber
\tilde{\Bd_k} = 
\begin{bmatrix}
d_1 + y^k - z_1^k \\
d_2 + y^k - z_2^k \\
\vdots \\
d_m + y^k - z_m^k \\
\end{bmatrix}, 
\quad \Bp_k = \begin{bmatrix}
\|\Bx^k\|^2 -\left(y^k\right)^2  \\
\|\Bx^k - \Ba_1\|^2 -\left(z_1^k\right)^2 \\
\vdots \\
\|\Bx^k - \Ba_m\|^2 -\left(z_m^k\right)^2 \\
\end{bmatrix}
\end{equation}
\begin{equation}
\nonumber
\BC_k = \begin{bmatrix}
-2\left(\Bx^k\right)^T & 2y^k & 0 & \hdots & 0 \\
-2\left(\Bx^k - \Ba_1\right)^T & 0 & 2z^k_1 & \hdots & 0 \\
\vdots & \vdots & \vdots & \ddots & \vdots \\
-2\left(\Bx^k - \Ba_m\right)^T & 0 & 0 & \hdots & 2z_m^k
\end{bmatrix},
\end{equation}

%\phantom{m}
\noindent
an solve
\setcounter{abc}{0}
\begin{eqnarray} 
\nonumber
\Min_{\Bdelta_x, \delta_y, \Bdelta_z, s}& &\| -\delta_y\symb{1_m} + \Bdelta_z - \tilde{\Bd_k} \|_2 + \mu_ks
\\ 
\nonumber
\mbox{subject to:}& &\BC_k\tilde{\Bdelta}  = \Bp_k 
\\
\nonumber
 & &|\tilde{\Bdelta}|  \leq \left(\beta + s\right)\symb{1_{m+3}} 
 \\
\nonumber
& & s \geq 0
\end{eqnarray}
\noindent
Denote the solution as $\tilde{\Bdelta}^k = (\Bdelta_x^*, \delta_y^*, \Bdelta_z^*)$. 

%\phantom{m}


3) Update  $\mu_{k+1} $ = min $(1.5\mu_k, \mu_{max})$, set $k = k+1$. Update $\tilde{\Bx}^{*}$ to
\setcounter{abc}{0}
\begin{eqnarray} 
\nonumber
\Bx^{*} = \Bx^k + \Bdelta_x^* \\
\nonumber
y^{*} = y^k + \delta_y^* \\
\nonumber
\Bz^{*} = \Bz^k + \Bdelta_z^*
\end{eqnarray}

%\phantom{m}


4) If $k = K_{max}$, terminate and output $\Bx^*$ as the solution; otherwise, set $\tilde{\Bx}^{k} = \tilde{\Bx}^{*}$  and repeat from Step 2. 

%\phantom{m}
%}
%}