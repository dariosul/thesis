\startchapter{Least Squares Localization by Sequential Convex Relaxation}
\label{chapter:socp}

%\textbf{Notes for further development.}
%
%2-step approach:
%
%1) identify outliers to reduce the error-prone data points
%
%2) apply the algorithm  to the redused data set
%
%Projection onto convex sets and projection onto rings \textit{cite GhStr}

\section{Second Order Cone Programming}

\section{Range-based localization}

Problem:
Given sensor array ${\Ba_i, i = 1, 2, ..., m}$ and noisey range measurements $r_i$ find the true \textit{unknown} location of $\Bx$ as 
\setcounter{abc}{0}
\begin{equation} \label{eq:5.1}
\Min \sum^m_i \left( \|\Bx - \Ba_i\| - r_i \right)^2 
\end{equation}
which can be (equivalently) written as 
\begin{eqnarray} \label{eq:5.2}
\setcounter{abc}{1}
\Min_{\Bx, \Bz}& &\sum^m_i \left( z_i - r_i \right)^2 \\
\setcounter{equation}{2}
\stepcounter{abc}
\mbox{subject to:}& &\|\Bx - \Ba_i\| = z_i, \quad i = 1, 2, ..., m
\end{eqnarray}
The constraint in \ref{eq:5.2} is hard to suffice, therefore we allow a relaxation:
\setcounter{abc}{0}
\begin{eqnarray} \label{eq:5.3}
\setcounter{abc}{1}
\Min_{\Bx, \Bz}& &\sum^m_i \left( z_i - r_i \right)^2 \\
\setcounter{equation}{3}
\stepcounter{abc}
\mbox{subject to:}& &\|\Bx - \Ba_i\| \leq (1+ \gamma)z_i  \\
\setcounter{equation}{3}
\stepcounter{abc}
& &\|\Bx - \Ba_i\| \geq (1 - \gamma)z_i, \quad i = 1, 2, ..., m
\end{eqnarray}
where $\gamma$ is small, typically $0 < \gamma < 0.5$. This would yield an approximate solution to \ref{eq:5.2} and therefore to \ref{eq:5.1}. 
By allowing $\gamma$ to sequentially/monotonically decrease from some small $0 < \gamma_0 < 0.5$ to 0 solution of \ref{eq:5.3} will converge to \ref{eq:5.2}.
\textit{Proof} Let $\gamma(k)$ be monotonically decteasing, where $k$ is an iteration count and $0 < \gamma_0 < 0.5$. Then 
$\lim_{\gamma \rightarrow 0} (1 + \gamma)z_i = z_i$ and $\lim_{\gamma \rightarrow 0} (1 - \gamma)z_i = z_i$. Therefore as $\gamma$ approaches 0, the feasible region of the problem in \ref{eq:5.3} will become equivalent to that in \ref{eq:5.2}.
As iterations proceed, the objective in \ref{eq:5.3} will not be monotonically decreasing but it will converge to the critical point.

Problem in \ref{eq:5.3} is nonconvex due to nonconvexity of one of its inequality constraint. The constraint in \ref{eq:5.3}b $\|\Bx - \Ba_i\| \leq (1+ \gamma)z_i$ is convex, the constraint in \ref{eq:5.3}c is not, because
\begin{equation}
\nonumber
\|\Bx - \Ba_i\| \geq (1 - \gamma)z_i \Longleftrightarrow \underbrace{-\|\Bx - \Ba_i\|}_{nonconvex} \leq -(1 - \gamma)z_i
\end{equation}
From convexity of the norm $\|\Bx - \Ba_i\|$ it follows that for some \textit{known} $\Bx_k$
\begin{equation}
\nonumber
\|\Bx - \Ba_i\| \geq \|\Bx_k - \Ba_i\| + \partial\|\Bx_k - \Ba_i\|^T(\Bx - \Ba_i)
\end{equation}
Hence the constraint in \ref{eq:5.3}c can be convexified by replacing it with its affine approximation
\begin{equation}
\nonumber
-\|\Bx_k - \Ba_i\| - \partial\|\Bx_k - \Ba_i\|^T(\Bx - \Ba_i) \leq -(1 - \gamma)z_i
\end{equation}
At the $k$th iteration when the iterate $\Bx_k$ is known, the nonconvex problem in \ref{eq:5.3} can be relaxed to an SOCP problem
\setcounter{abc}{0}
\begin{eqnarray} \label{eq:5.4}
\stepcounter{abc}
\Min_{\Bx, \Bz}& &\sum^m_i \left( z_i - r_i \right)^2 \\
\setcounter{equation}{4}
\stepcounter{abc}
\mbox{subject to:}& &\|\Bx - \Ba_i\|  \leq  (1+ \gamma)z_i  
\end{eqnarray}
\begin{equation}
\setcounter{equation}{4}
\stepcounter{abc}
\qquad \qquad \qquad  -\|\Bx_k - \Ba_i\| - \partial\|\Bx_k -\Ba_i\|^T(\Bx - \Ba_i)  \leq  -(1 - \gamma)z_i, \quad i = 1, 2, ..., m
\end{equation}
\setcounter{abc}{0}
The relaxation parameter $\gamma$ controls the size of the convex hull that defines a feasibility region of the problem \ref{eq:5.4}.
$\gamma$ needs to be monotonically decreasing with increase of the iteration count. Start with some $0 < \gamma_0 < 0.5$, typically $\gamma_0 = 0.3$ or 0.2 is good. After $k$th iteration update $\gamma_{k+1}$ linearly as
\begin{equation}
\nonumber
\gamma_{k+1} = \gamma_0 - k\frac{\gamma_0}{K_{max} - 1}
\end{equation}
or quadratically as
\begin{equation}
\nonumber
\gamma_{k+1} = \gamma_0\frac{(K_{max} - 1 - k)^2}{(K_{max} - 1)^2}
\end{equation}


%\startchapter{SOCP}

\section{Range-Difference Localization}

\subsection{Problem Statement}

Measurement model
\begin{equation} \label{eq:6.1}
d_i = \|\Bx - \Ba_i\| - \|\Bx\| + noise
\end{equation}

Least-squares formulation
\begin{equation} \label{eq:6.2}
\Min \sum^m_{i=1}\left( \|\Bx - \Ba_i\| - \|\Bx\| - d_i\right)^2
\end{equation}

\subsection{Sequential Relaxation}
The problem in \ref{eq:6.2} can be equivalently written as
\begin{eqnarray} \label{eq:6.3}
\Min \sum^m_{i=1}\left( z_i - y - d_i\right)^2\\
\nonumber
\mbox{subject to: } \|\Bx - \Ba_i\| = z_i \\
\nonumber
\|\Bx\|  = y, \quad  i = 1, 2, \ldots m
\end{eqnarray}
Let $\tilde{\Bx} = [\Bx^T \ y \ z_1 \ldots z_m]^T$, $\Bx \in R^n, y \in R, \Bz \in R^m$ be a known feasible point of the problem in \ref{eq:6.3}. Let $\tilde{\Bdelta} = [\Bdelta_x^T \  \delta_y \ \delta_{z_1} \ \ldots \  \delta_{z_m}]^T$, $\Bdelta_x \in R^n, \Bdelta_y \in R, \Bdelta_z \in R^m$ is a small perturbation to it, such that $|\tilde{\Bdelta}| \leq \beta\symb{1_{m+3}}$ and $\beta > 0$ is a small positive constant. We need to find an increment vector $\tilde{\Bdelta} = [\Bdelta_x^T \  \delta_y \ \Bdelta_{z}^T]^T$ such that the next iterate
\begin{eqnarray} \label{eq:6.4}
\Bx^{k+1} = \Bx^k + \Bdelta_x \\
\nonumber
y^{k+1} = y^k + \delta_y \\
\nonumber
\Bz^{k+1} = \Bz^k + \Bdelta_z
\end{eqnarray}
remains strictly feasible. 
At the $k$th iterations with $\tilde{\Bx}^k$ \textit{known} update $(\Bx^k, y^k, \Bz^k)$ to 
Substituting \ref{eq:6.4} in \ref{eq:6.3} the objective in \ref{eq:6.3}a can be written as
\begin{eqnarray} \label{eq:6.5}
F(\hat{\Bx}) & =  & \sum^m_{i = 1} \left(z^k_i + \delta_{z_i} - (y^k + \delta_y) - d_i \right)^2 \\
\nonumber
& = & \sum^m_{i = 1} \left(- \delta_y + \delta_{z_i}  - \tilde{d_i^k} \right)^2
\end{eqnarray}
where 
\begin{equation}
\nonumber
\tilde{d}^k_i =  d_i - y^k - z_i^k
\end{equation}
 are grouped known constant terms.
Substituting \ref{eq:6.4}b in \ref{eq:6.3}b
\begin{equation}
\nonumber
\|\Bx^k + \Bdelta^k_x - \Ba_i\| = z^k_i +\delta_{z_i}, \quad i = 1, 2, \ldots, m
\end{equation}
The constraints can be convexified by squaring both sides of the equality and then re-grouping the terms on the left-hand side as
\begin{eqnarray}
\nonumber
\|(\Bx^k  - \Ba_i) + \Bdelta^k_x\|^2 & = & \left(z^k_i +\delta_{z_i}\right)^2 \\
\nonumber
\Leftrightarrow 
\|\Bx^k  - \Ba_i\|^2 + 2\left( \Bx^k  - \Ba_i \right)^T\Bdelta_x + \cancelto{\mbox{\scriptsize{0}}}{\|\Bdelta_x\|^2}  & = & \left(z_i^k\right)^2 + 2z_i^k\delta_{z_i} + \cancelto{\mbox{\scriptsize{0}}}{\Bdelta^2_{z_i}} \\
\nonumber
\Leftrightarrow \|\Bx^k  - \Ba_i\|^2 + 2\left( \Bx^k  - \Ba_i \right)^T\Bdelta_x   & \approx & \left(z_i^k\right)^2 + 2z_i^k\delta_{z_i} \\
\nonumber
\Leftrightarrow \|\Bx^k  - \Ba_i\|^2 + 2\left( \Bx^k  - \Ba_i \right)^T\Bdelta_x   & \approx & \left(z_i^k\right)^2 + 2z_i^k\delta_{z_i} \\
\nonumber
- 2\left( \Bx^k  - \Ba_i \right)^T\Bdelta_x + 2z_i^k\delta_{z_i}  & \approx & \|\Bx^k  - \Ba_i\|^2 - \left(z_i^k\right)^2
\end{eqnarray}
Repeating the similar procedure with the constraint in \ref{eq:6.3}c
\begin{eqnarray}
\nonumber
\|\Bx^k + \Bdelta_x\| & = & y^k + \delta_y \\
\nonumber
\|\Bx^k + \Bdelta_x\|^2 & = & \left(y^k + \delta_y \right)^2 \\
\nonumber
\Leftrightarrow \|\Bx^k\|^2 + 2\Bdelta_x^T\Bx^k + \cancelto{\mbox{\scriptsize{0}}}{\|\Bdelta_x\|^2} & = & \left(y^k\right)^2 + 2y^k\delta_y +  \cancelto{\mbox{\scriptsize{0}}}{\delta_y^2} \\
\nonumber
\Leftrightarrow -2\left(\Bx^k\right)^T\delta_x + 2y^k\delta & \approx & \|\Bx\|^2 - \left(y^k\right)^2
\end{eqnarray}
The problem in \ref{eq:6.3} can now be written in terms of the \textit{known} feasible  iterate $\tilde{\Bx}^k$ and its \textit{unknown} increment  $\tilde{\Bdelta} = [\Bdelta_x^T \  \delta_y \ \Bdelta_{z}^T]^T$ as
\begin{eqnarray} \label{eq:6.6}
\Min_{\Bdelta_x, \delta_y, \Bdelta_z}& &\sum^m_{i=1}\left( -\delta_y + \delta_{z_i} -\tilde{d_i}^k\right)^2\\
\nonumber
\mbox{subject to:}& &- 2\left( \Bx^k  - \Ba_i \right)^T\Bdelta_x + 2z_i^k\delta_{z_i}  = \|\Bx^k  - \Ba_i\|^2 - \left(z_i^k\right)^2 \\
\nonumber
& &-2\left(\Bx^k\right)^T\delta_x + 2y^k\delta = \|\Bx^k\|^2 - \left(y^k\right)^2 \\
\nonumber
& &|{\tilde{\Bdelta}}|  \leq \beta\symb{1_{m+3}}, \quad  i = 1, 2, \ldots m
\end{eqnarray}
It is obvious to see that the  problem in \ref{eq:6.6} can be written in the following form
\begin{eqnarray} \label{eq:6.7}
\Min_{{\Bdelta_x, \delta_y, \Bdelta_z}}& &\| -\delta_y\symb{1_m} + \Bdelta_z - \tilde{\Bd}^k \|_2 
\\ \nonumber
\mbox{subject to:}& &\BC_k\tilde{\Bdelta}  = \Bp_k \\
\nonumber
 & &|\tilde{\Bdelta}|  \leq \beta \symb{1_{m+3}}
\end{eqnarray}
where
\begin{equation} \label{eq:6.8}
\tilde{\Bd^k} = 
\begin{bmatrix}
d_1 + y^k - z_1^k \\
d_2 + y^k - z_2^k \\
\vdots \\
d_m + y^k - z_m^k \\
\end{bmatrix}, 
\quad \Bp_k = \begin{bmatrix}
\|\Bx^k\|^2 -\left(y^k\right)^2  \\
\|\Bx^k - \Ba_1\|^2 -\left(z_1^k\right)^2 \\
\vdots \\
\|\Bx^k - \Ba_m\|^2 -\left(z_m^k\right)^2 \\
\end{bmatrix}
\end{equation}
\begin{equation}
\nonumber
\BC_k = \begin{bmatrix}
-2\left(\Bx^k\right)^T & 2y^k & 0 & \hdots & 0 \\
-2\left(\Bx^k - \Ba_1\right)^T & 0 & 2z^k_1 & \hdots & 0 \\
\vdots & \vdots & \vdots & \ddots & \vdots \\
-2\left(\Bx^k - \Ba_m\right)^T & 0 & 0 & \hdots & 2z_m^k
\end{bmatrix},
\quad \tilde{\Bdelta} = \begin{bmatrix}
\Bdelta_x \\
\delta_y \\
\Bdelta_z
\end{bmatrix}
\end{equation}
A technical problem making the formulation in (3.14) difficult to implement is that it requires a feasible initial point $\Bx^k$. The iterate $\tilde{\Bx}^k$ is strictly feasible and known but it is not guaranteed that the $\tilde{\Bx}^{k+1}$ will also be feasible. The problem can be overcome by introducing nonnegative slack variable $s$ into the constraints in \ref{eq:6.8}c to replace their right-hand sides by relaxed upper bounds (as these new bounds themselves are nonnegative variables). To allow non-feasible increments $\tilde{\Bdelta}$, the problem can be overcome by introducing a
nonnegative slack variable $s \geq 0$ \ref{eq:6.8}c to replace their right-hand sides  by a  relaxed upper bound (as this new bound itself is a nonnegative variable). We allow the constraint in $|\tilde{\Bdelta}|  \leq \beta \symb{1_{m+3}}$ be violated so that the  region with lower value of the objective function can be found. 
This leads to a following sequential relaxation of the problem in \ref{eq:6.3}
\begin{eqnarray} \label{eq:6.9}
\Min_{\Bdelta_x, \delta_y, \Bdelta_z, s}& &\| -\delta_y\symb{1_m} + \Bdelta_z - \tilde{\Bd}^k \|_2 + \mu_ks
\\ \nonumber
\mbox{subject to:}& &\BC_k\tilde{\Bdelta}  = \Bp_k 
\\
\nonumber
 & &|\tilde{\Bdelta}|  \leq \left(\beta + s\right)\symb{1_{m+3}} 
 \\
\nonumber
& & s \geq 0
\end{eqnarray}
where the weight $\mu_k \geq 0$ increases as iterations proceed until it reaches an upper limit
$\mu_{max}$. By using a monotonically increasing $\mu_k$ for the penalty term in \ref{eq:6.9}a, the
algorithm reduces the slack variables $s$  very quickly. As a result, new iterates
quickly become feasible as $s$  vanishes. The upper limit $\mu_{max}$ is imposed to avoid
numerical difficulties that may occur if $\mu_{k}$ becomes too large and to ensure convergence if a feasible region is not found.


\subsection{The Algorithm}
The constraint $\beta$ was imposed on each element of the vector $\tilde{\Bdelta}$ to guarantee that at each iteration is sufficiently small.


Dropping the constraints in \ref{eq:6.8}f,g allows more variety in choosing the search direction, which increases the likelihood of the algorithm not to get trapped in the local minimimum.

\subsection{Numerical Results}